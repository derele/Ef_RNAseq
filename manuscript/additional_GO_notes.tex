\documentclass[a4paper, 11pt]{article}
\usepackage{comment} % enables the use of multi-line comments (\ifx \fi) 
\usepackage{fullpage} % changes the margin

\begin{document}
%Header-Make sure you update this information!!!!
\noindent
\large\textbf{Below are (old) notes on GO and specific genes, which can be merged in discussion above some details might also have changed due to re-analysis} \hfill \textbf{Totta Kasemo} \\

\section*{Preparation for invasion in oocysts}
The mRNA profile in the oocyst stage is mainly determined by highly
abundant genes in cluster 4.  Overrepresented GO-terms in this cluster
are enriched by ortholog genes to peptidases, microneme localized
proteins reported to be involved in invasion, genes associated with
adhesion in protozoans (and with clotting in higher eukaryotes), and
genes that are annotated to be involved in amino acid biosynthesis.
Aminopeptidase N ('related' annotation) is the reported ortholog for
three genes with abundant mRNAs in oocysts. In humans, this enzyme has
been reported to cleave peptides bound to major histocompatibility
complex, MHC, II (UniProt reference if we want to keep this... but
does any secretion happen from oocysts...? Or is this too far-fetched
to be interesting?).

A Thrombospondin type 1 domain-containing protein ortholog is highly
abundant in cluster 4 (high abundance in oocysts). 
%%% This sentence used above:
%%%Thrombospondin type
%%%1 domains have been reported in \textit{E. tenella} microneme
%%%localizing proteins, MIC, e.g. MIC4 (Tomley01) 
%In \textit{E. tenella}
%MIC4 mRNA was reported in sporozoites where it localizes to the apical
%end, and in late schizonts and late oocyst stages, when sporozoites
%are forming. (Tomley01). 

For the same gene, the \textit{T. gondii}
annotation is Sushi domain-containing protein, which is also the
ortholog annotation of another gene in this cluster. In the related
malaria parasite \textit{P. falciparum} the apical sushi protein, ASP,
(which has a sushi domain) localizes to micronemes in merozoites but
not other stages (OKeeffe05).  Limulus clotting factor C, Coch-5b2
(Cochlin) and Lgl1, LCCL, (syn. F5/8 domain) domains are associated
discoidin lectin domains andhereby with adhesion.  (Pfam entries for
'LCCL domain' and 'Discoidin domain', May 2016).  Taken together, this
indicates that the thrombospondin and sushi-domain genes
(EfaB\_MINUS\_4114.g412 and EfaB\_PLUS\_1425.g183) are involved in
sporozoite invasion in \textit{E. falciformis} and that the mRNAs are
transcribed and available before excystation.  A role in merozoite
re-invasion in \textit{E. falciformis} is not indicated by our data.
The LCCL domain annotation and thrombospondins role in higher
eukaryotes also indicates that adhesion or preparation for adhesion is
important in oocysts.  We suggest (speculate...?) that the
thrombospondin annotated ortholog and the LCCL domain-containing
protein (EfaB\_MINUS\_11233.g986) are involved in cell adhesion in
\textit{E. falciformis}.
%One examples is found in the slime mold \textit{Dictyostelium
%discoideum)} in which the LCCL domain is part of the mold's discoidin
%adhesion protein. (Pfam entry for 'discoidin domain', May 2016)

\section*{Amino acid biosynthesis in oocysts}
High abundance of aminotransferase mRNAs indicate amino acid
biosynthesis or preparation for the same in oocysts (cluster 4).  We
identify D-3-phosphoglycerate dehydrogenase and alanine dehydrogenase
orthologs, which are enzymes contributing to L-serine and L-alanine
production, respectively. A putative \textit{Eimeria}
spp. cystathionine beta-synthase, CBS, in this cluster also indicates
de novo cysteine production. Alkyl sulfatase mRNA is also abundant in
oocysts. Generally, this enzyme enables an organism to exploit organic
sulfur to produce and incorporate inorganic sulfur into the amino
acids cysteine and methionine, when no inorganoc sulfur is available.

'Embryonic development' Nicalin 1, patched family protein (hedgehog)


\section*{Oocysts contain mRNA for fatty acid catabolism}
MmgE/PrpD is overrepresented in oocysts. The enzyme is important for
propionate catabolism in the 2-methylcitric acid cycle and has been
shown to be used by the intestinal intracellular bacterium
\textit{Salmonella typhimurium} to generate pyruvate (Horswill99).
Propionate is one of two most abundant small-chain fatty acids in the
gut along with butyrate. Both fatty acids are largely produced as
degradation products from food by commensal bacteria (Sun13). Sharing
the intestines as a niche with \textit{S. typhimurium} it is possible
that also \textit{E. falciformis} uses Mmg/PrpD to exploit available
propionate for pyruvate production.

\section*{Oocyst highly abundant mRNAs are downregulated in sporozoites}
Interestingly, the genes described above which are thought to be
involved in amino acid biosynthesis and invasion are highly abundant
in oocysts but are underrepresented in schizont stages (day 3 and day
5 samples) and even in sporozoites. An average abundance was detected
on day 7 for these genes, indicating a role in either gametes or early
oocyst formation.  This pattern supports the suggestion that these
specific mRNAs (cluster 4) for invasion and biosynthetic processes are
prepared (and possibly expressed) in the oocyst stage but are no
longer detectable in the cell at the timepont when the protein is
assumed to be in use (sporozoites and merozoite stages).  Therefore,
correlating mRNA prevalence with biological function at the timepoint
when mRNAs are detected must be done with care.

\section*{Down in sporozoites and oocysts -->  cluster 3..........}
Oocysts: profile for 3, 5, 6:

Sporozoites: 3 and 5

Day 7: 5 and 7

Specific genes....  Enolase 2, encoded by Eno2, is among the
downreted genes in oocysts and sporozoites.  In \textit{T. gondii}
the paralog Eno1 is strongly associated with the cyst (bradyzoite)
stage and Eno2 is associated with tachyzoite stages (kibe05). It is
therefore expected that this mRNA is underrepresented in oocysts and
our data also show that the same is true in sporozoites for
\textit{E. falciformis}. (TK: If important we could look specifically
for Eno1 in cluster 4).

%%%%%%%%%%%%%%%%%%%%%%%%%%%%%%%%%%%%%%%%%%%%%%%%%%%%%%%%%%%%%%%%%%%%%%%%%%%%%%%%%%%%%%%%%%
%%%%%%%%%%%%%%%%%%%%%%% DAY 7  %%%%%%%%%%%%%%%%%%%%%%%%%%%%%%%%%%%%%%%%%%%%%%%%%
\section*{Motility-related mRNAs indicate gamete development on day 7}
Two clusters contain genes with mRNAs highly abundant on day 7 p.i;
cluster 1 and 2.  Dynein, kinesin and tubulin are annotations highly
represented among orthologs of genes in both these clusters. The
annotations indicate an important role for motility at this timepoint,
probably reflecting development of microgametes.  In addition, in
cluster 2, there are two 'EF-hand domain containing proteins'
annotations as well as caltractin, centrin-1, and troponin
annotations. Caltractin and centrin-1 are associated with the
centrosome and structure and function of microtubuli in mammals, and
troponin is linked to muscle function (UniProt).  Also potentially
linked to motility is the occurence of growth arrest specific protein
8, Gas8, which in the mouse has been reported to be highly expressed
in the testes and important for mouse sperm function (Yeh02).

Other genes among the 38 indicate carbon fixation
(glycolysis/gluconeogenesis) or conversions of nucleoside
phosphates. In addition, a Ras family protein, RNA polymerase II
transcription initiation factor and Sec23 and Sec24 were among
orthologs identified in \textit{E. falciformis} cluster 2.

In cluster 1, carbon metabolism genes are represented by
6-phosphogluconate dehydrogenase and glycogen phosphorylase family
protein 1. UDP-glucose 4-epimerase and amiloride-sensitive amine
oxidase are reported as upregulated in gametocytes in
\textit{E. tenella} by RNA-seq (Walker15) and suggested by those
authors to play a role in cyst wall synthesis.

\section*{Microneme proteins highly expressed on day 7 p.i.}
Unintuitively for a protozoan organism, seven out of eight GO
biological process terms in cluster 1 are associated with wound
healing and blood coagulation.  An explanation is offered by some of
the orthologs to the three \textit{E. falciformis} genes responsible
for these terms. In protozoa, e.g., other \textit{Eimeria} spp. and
\textit{Toxoplasma gondii} orthologs are annotated as 'Micronemal
protein MIC4, related' (E. tenella) and more generally for several
other protozoa, 'PAN domain containing proteins'.  The PAN domain is
found in the plasminogen/hepatocyte growth factor family and in
coagulation factor XI family (REF), explaining why terms related to
blood coagulation are enriched by these genes.  Later publications on
\textit{T. gondii} (Marchant12) also associate PAN domains and
proteins in apicomplexan parasites with micronemes and therefore
invasion. In our case, this is peculiar, since the enrichment appears
on day 7 p.i.. A possible role at this timepoint is suggested by work
on the fungi \textit{Sclerotinia sclerotiorum} where Yu et
al. reported an important role for PAN domain proteins in cell wall
integrity (Yu12). This role for MIC proteins has to our knowledge not
been investigated in apicomplexan parasites.  The PAN domain domain
has also been reported to be common in nematodes such as
\textit{Cahabditis elegans}, however the function is not
understood. (Thordai99) The other two GO terms in the cluster of day
seven upregulated genes are DNA replication and DNA replication
initiation, which most likely reflects late stage schizogony or gamete
formation.  Six genes contribute to this enrichment and orthologs are
either annotated as DNA replcation licencing factors, DNA polymerases
or minichromosome maintenance proteins 2/3/5/7, Mcm2/3/5/7.

\end{document}

