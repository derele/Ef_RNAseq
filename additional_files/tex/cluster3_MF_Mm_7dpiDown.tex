% latex table generated in R 3.2.2 by xtable 1.8-2 package
% Fri Apr  8 11:17:38 2016
\begin{table}[ht]
\centering
\begin{tabular}{rllrrrrr}
  \hline
 & GO.ID & Term & Annotated & Significant & Expected & p.value & adj.p \\ 
  \hline
1 & GO:0015020 & glucuronosyltransferase activity &  55 &  16 & 2.00 & 0.00 & 0.00 \\ 
  2 & GO:0008194 & UDP-glycosyltransferase activity & 106 &  18 & 3.85 & 0.00 & 0.00 \\ 
  3 & GO:0016878 & acid-thiol ligase activity &  15 &   7 & 0.54 & 0.00 & 0.00 \\ 
  4 & GO:0048037 & cofactor binding & 208 &  23 & 7.56 & 0.00 & 0.00 \\ 
  5 & GO:0042803 & protein homodimerization activity & 538 &  42 & 19.54 & 0.00 & 0.00 \\ 
  6 & GO:0046983 & protein dimerization activity & 797 &  55 & 28.95 & 0.00 & 0.00 \\ 
  7 & GO:0003857 & 3-hydroxyacyl-CoA dehydrogenase activity &   8 &   5 & 0.29 & 0.00 & 0.00 \\ 
  8 & GO:0016758 & transferase activity, transferring hexos... & 142 &  18 & 5.16 & 0.00 & 0.00 \\ 
  9 & GO:0016877 & ligase activity, forming carbon-sulfur b... &  28 &   8 & 1.02 & 0.00 & 0.00 \\ 
  10 & GO:0015645 & fatty acid ligase activity &  14 &   6 & 0.51 & 0.00 & 0.00 \\ 
  11 & GO:0004364 & glutathione transferase activity &  24 &   7 & 0.87 & 0.00 & 0.00 \\ 
  12 & GO:0016757 & transferase activity, transferring glyco... & 208 &  21 & 7.56 & 0.00 & 0.00 \\ 
  13 & GO:0016405 & CoA-ligase activity &  11 &   5 & 0.40 & 0.00 & 0.00 \\ 
  14 & GO:0016421 & CoA carboxylase activity &   6 &   4 & 0.22 & 0.00 & 0.00 \\ 
  15 & GO:0016616 & oxidoreductase activity, acting on the C... &  92 &  13 & 3.34 & 0.00 & 0.00 \\ 
  16 & GO:0051287 & NAD binding &  46 &   9 & 1.67 & 0.00 & 0.00 \\ 
  17 & GO:0016491 & oxidoreductase activity & 523 &  38 & 19.00 & 0.00 & 0.00 \\ 
  18 & GO:0003824 & catalytic activity & 4065 & 187 & 147.66 & 0.00 & 0.00 \\ 
  19 & GO:0016614 & oxidoreductase activity, acting on CH-OH... & 110 &  14 & 4.00 & 0.00 & 0.00 \\ 
  20 & GO:0016885 & ligase activity, forming carbon-carbon b... &   7 &   4 & 0.25 & 0.00 & 0.00 \\ 
  21 & GO:0031406 & carboxylic acid binding & 128 &  15 & 4.65 & 0.00 & 0.00 \\ 
  22 & GO:0043177 & organic acid binding & 129 &  15 & 4.69 & 0.00 & 0.00 \\ 
  23 & GO:0016712 & oxidoreductase activity, acting on paire... &  14 &   5 & 0.51 & 0.00 & 0.00 \\ 
  24 & GO:0033293 & monocarboxylic acid binding &  41 &   8 & 1.49 & 0.00 & 0.00 \\ 
  25 & GO:0070330 & aromatase activity &   8 &   4 & 0.29 & 0.00 & 0.00 \\ 
  26 & GO:0005496 & steroid binding &  53 &   9 & 1.93 & 0.00 & 0.00 \\ 
  27 & GO:0038023 & signaling receptor activity & 361 &  28 & 13.11 & 0.00 & 0.00 \\ 
  28 & GO:0022838 & substrate-specific channel activity & 139 &  15 & 5.05 & 0.00 & 0.00 \\ 
  29 & GO:0050662 & coenzyme binding & 157 &  16 & 5.70 & 0.00 & 0.00 \\ 
  30 & GO:0004467 & long-chain fatty acid-CoA ligase activit... &   9 &   4 & 0.33 & 0.00 & 0.00 \\ 
  31 & GO:0004774 & succinate-CoA ligase activity &   9 &   4 & 0.33 & 0.00 & 0.00 \\ 
  32 & GO:0015267 & channel activity & 146 &  15 & 5.30 & 0.00 & 0.00 \\ 
  33 & GO:0022803 & passive transmembrane transporter activi... & 146 &  15 & 5.30 & 0.00 & 0.00 \\ 
  34 & GO:0016831 & carboxy-lyase activity &  26 &   6 & 0.94 & 0.00 & 0.00 \\ 
  35 & GO:0016830 & carbon-carbon lyase activity &  37 &   7 & 1.34 & 0.00 & 0.00 \\ 
  36 & GO:0008144 & drug binding &  76 &  10 & 2.76 & 0.00 & 0.00 \\ 
  37 & GO:0004872 & receptor activity & 448 &  31 & 16.27 & 0.00 & 0.00 \\ 
  38 & GO:0042802 & identical protein binding & 915 &  53 & 33.24 & 0.00 & 0.00 \\ 
  39 & GO:0004075 & biotin carboxylase activity &   5 &   3 & 0.18 & 0.00 & 0.00 \\ 
  40 & GO:0005007 & fibroblast growth factor-activated recep... &   5 &   3 & 0.18 & 0.00 & 0.00 \\ 
  41 & GO:0004888 & transmembrane signaling receptor activit... & 294 &  23 & 10.68 & 0.00 & 0.00 \\ 
  42 & GO:0022836 & gated channel activity &  97 &  11 & 3.52 & 0.00 & 0.00 \\ 
  43 & GO:1901681 & sulfur compound binding & 129 &  13 & 4.69 & 0.00 & 0.00 \\ 
  44 & GO:0001972 & retinoic acid binding &   6 &   3 & 0.22 & 0.00 & 0.00 \\ 
  45 & GO:0022857 & transmembrane transporter activity & 468 &  31 & 17.00 & 0.00 & 0.00 \\ 
  46 & GO:0005216 & ion channel activity & 135 &  13 & 4.90 & 0.00 & 0.00 \\ 
  47 & GO:0004553 & hydrolase activity, hydrolyzing O-glycos... & 119 &  12 & 4.32 & 0.00 & 0.00 \\ 
  48 & GO:0001517 & N-acetylglucosamine 6-O-sulfotransferase... &  14 &   4 & 0.51 & 0.00 & 0.00 \\ 
  49 & GO:0004027 & alcohol sulfotransferase activity &  14 &   4 & 0.51 & 0.00 & 0.00 \\ 
  50 & GO:0004394 & heparan sulfate 2-O-sulfotransferase act... &  14 &   4 & 0.51 & 0.00 & 0.00 \\ 
  51 & GO:0016232 & HNK-1 sulfotransferase activity &  14 &   4 & 0.51 & 0.00 & 0.00 \\ 
  52 & GO:0017095 & heparan sulfate 6-O-sulfotransferase act... &  14 &   4 & 0.51 & 0.00 & 0.00 \\ 
  53 & GO:0018721 & trans-9R,10R-dihydrodiolphenanthrene sul... &  14 &   4 & 0.51 & 0.00 & 0.00 \\ 
  54 & GO:0018722 & 1-phenanthrol sulfotransferase activity &  14 &   4 & 0.51 & 0.00 & 0.00 \\ 
  55 & GO:0018723 & 3-phenanthrol sulfotransferase activity &  14 &   4 & 0.51 & 0.00 & 0.00 \\ 
  56 & GO:0018724 & 4-phenanthrol sulfotransferase activity &  14 &   4 & 0.51 & 0.00 & 0.00 \\ 
  57 & GO:0018725 & trans-3,4-dihydrodiolphenanthrene sulfot... &  14 &   4 & 0.51 & 0.00 & 0.00 \\ 
  58 & GO:0018726 & 9-phenanthrol sulfotransferase activity &  14 &   4 & 0.51 & 0.00 & 0.00 \\ 
  59 & GO:0018727 & 2-phenanthrol sulfotransferase activity &  14 &   4 & 0.51 & 0.00 & 0.00 \\ 
  60 & GO:0019111 & phenanthrol sulfotransferase activity &  14 &   4 & 0.51 & 0.00 & 0.00 \\ 
  61 & GO:0034483 & heparan sulfate sulfotransferase activit... &  14 &   4 & 0.51 & 0.00 & 0.00 \\ 
  62 & GO:0034930 & 1-hydroxypyrene sulfotransferase activit... &  14 &   4 & 0.51 & 0.00 & 0.00 \\ 
  63 & GO:0050698 & proteoglycan sulfotransferase activity &  14 &   4 & 0.51 & 0.00 & 0.00 \\ 
  64 & GO:0051922 & cholesterol sulfotransferase activity &  14 &   4 & 0.51 & 0.00 & 0.00 \\ 
  65 & GO:0080131 & hydroxyjasmonate sulfotransferase activi... &  14 &   4 & 0.51 & 0.00 & 0.00 \\ 
  66 & GO:0004658 & propionyl-CoA carboxylase activity &   2 &   2 & 0.07 & 0.00 & 0.00 \\ 
  67 & GO:0004776 & succinate-CoA ligase (GDP-forming) activ... &   2 &   2 & 0.07 & 0.00 & 0.00 \\ 
  68 & GO:0016880 & acid-ammonia (or amide) ligase activity &   2 &   2 & 0.07 & 0.00 & 0.00 \\ 
  69 & GO:1902282 & voltage-gated potassium channel activity... &   2 &   2 & 0.07 & 0.00 & 0.00 \\ 
  70 & GO:0022891 & substrate-specific transmembrane transpo... & 422 &  28 & 15.33 & 0.00 & 0.00 \\ 
  71 & GO:0016798 & hydrolase activity, acting on glycosyl b... & 140 &  13 & 5.09 & 0.00 & 0.00 \\ 
  72 & GO:0016765 & transferase activity, transferring alkyl... &  50 &   7 & 1.82 & 0.00 & 0.00 \\ 
  73 & GO:0046906 & tetrapyrrole binding &  64 &   8 & 2.32 & 0.00 & 0.00 \\ 
  74 & GO:0000252 & C-3 sterol dehydrogenase (C-4 sterol dec... &  16 &   4 & 0.58 & 0.00 & 0.00 \\ 
  75 & GO:0001537 & N-acetylgalactosamine 4-O-sulfotransfera... &  16 &   4 & 0.58 & 0.00 & 0.00 \\ 
  76 & GO:0004495 & mevaldate reductase activity &  16 &   4 & 0.58 & 0.00 & 0.00 \\ 
  77 & GO:0008395 & steroid hydroxylase activity &  16 &   4 & 0.58 & 0.00 & 0.00 \\ 
  78 & GO:0008875 & gluconate dehydrogenase activity &  16 &   4 & 0.58 & 0.00 & 0.00 \\ 
  79 & GO:0018451 & epoxide dehydrogenase activity &  16 &   4 & 0.58 & 0.00 & 0.00 \\ 
  80 & GO:0018452 & 5-exo-hydroxycamphor dehydrogenase activ... &  16 &   4 & 0.58 & 0.00 & 0.00 \\ 
  81 & GO:0018453 & 2-hydroxytetrahydrofuran dehydrogenase a... &  16 &   4 & 0.58 & 0.00 & 0.00 \\ 
  82 & GO:0019152 & acetoin dehydrogenase activity &  16 &   4 & 0.58 & 0.00 & 0.00 \\ 
  83 & GO:0032442 & phenylcoumaran benzylic ether reductase ... &  16 &   4 & 0.58 & 0.00 & 0.00 \\ 
  84 & GO:0032866 & D-xylose:NADP reductase activity &  16 &   4 & 0.58 & 0.00 & 0.00 \\ 
  85 & GO:0032867 & L-arabinose:NADP reductase activity &  16 &   4 & 0.58 & 0.00 & 0.00 \\ 
  86 & GO:0033709 & D-arabinitol dehydrogenase, D-ribulose f... &  16 &   4 & 0.58 & 0.00 & 0.00 \\ 
  87 & GO:0034831 & (R)-(-)-1,2,3,4-tetrahydronaphthol dehyd... &  16 &   4 & 0.58 & 0.00 & 0.00 \\ 
  88 & GO:0034840 & 3-hydroxymenthone dehydrogenase activity &  16 &   4 & 0.58 & 0.00 & 0.00 \\ 
  89 & GO:0035380 & very long-chain-3-hydroxyacyl-CoA dehydr... &  16 &   4 & 0.58 & 0.00 & 0.00 \\ 
  90 & GO:0043713 & (R)-2-hydroxyisocaproate dehydrogenase a... &  16 &   4 & 0.58 & 0.00 & 0.00 \\ 
  91 & GO:0044103 & L-arabinose 1-dehydrogenase (NADP+) acti... &  16 &   4 & 0.58 & 0.00 & 0.00 \\ 
  92 & GO:0044105 & L-xylulose reductase (NAD+) activity &  16 &   4 & 0.58 & 0.00 & 0.00 \\ 
  93 & GO:0048258 & 3-ketoglucose-reductase activity &  16 &   4 & 0.58 & 0.00 & 0.00 \\ 
  94 & GO:0052677 & D-arabinitol dehydrogenase, D-xylulose f... &  16 &   4 & 0.58 & 0.00 & 0.00 \\ 
  95 & GO:0016829 & lyase activity & 144 &  13 & 5.23 & 0.00 & 0.00 \\ 
  96 & GO:0003996 & acyl-CoA ligase activity &   8 &   3 & 0.29 & 0.00 & 0.00 \\ 
  97 & GO:0010435 & 3-oxo-2-(2'-pentenyl)cyclopentane-1-octa... &   8 &   3 & 0.29 & 0.00 & 0.00 \\ 
  98 & GO:0018854 & 3-isopropenyl-6-oxoheptanoyl-CoA synthet... &   8 &   3 & 0.29 & 0.00 & 0.00 \\ 
  99 & GO:0018855 & 2-oxo-delta3-4,5,5-trimethylcyclopenteny... &   8 &   3 & 0.29 & 0.00 & 0.00 \\ 
  100 & GO:0018856 & benzoyl acetate-CoA ligase activity &   8 &   3 & 0.29 & 0.00 & 0.00 \\ 
   \hline
\end{tabular}
\end{table}
