%%%%%%%%%%%%%%%%%%%%%%%%%%%%%%%%%%%%%%%%%
% University/School Laboratory Report
% LaTeX Template
% Version 3.1 (25/3/14)
%
% This template has been downloaded from:
% http://www.LaTeXTemplates.com
%
% Original author:
% Linux and Unix Users Group at Virginia Tech Wiki 
% (https://vtluug.org/wiki/Example_LaTeX_chem_lab_report)
%
% License:
% CC BY-NC-SA 3.0 (http://creativecommons.org/licenses/by-nc-sa/3.0/)
%
%%%%%%%%%%%%%%%%%%%%%%%%%%%%%%%%%%%%%%%%%

%%%%%%%%%%%%%%%%%%%%%%
%% Totta:
%% This collection aims to provide an overview of the enriched GO-terms in the
%% different clusters in the heatmaps. 
%% We include both mouse and Eimeria data

%%%%%%%%%%%%%%%%%%%%%%

%----------------------------------------------------------------------------------------
%	PACKAGES AND DOCUMENT CONFIGURATIONS
%----------------------------------------------------------------------------------------

\documentclass{article}

%\usepackage[version=3]{mhchem} % Package for chemical equation typesetting
%\usepackage{siunitx} % Provides the \SI{}{} and \si{} command for typesetting SI units
\usepackage{graphicx} % Required for the inclusion of images
\usepackage[section]{placeins}
\usepackage[hscale=0.8,vscale=0.8]{geometry}
%\usepackage{natbib} % Required to change bibliography style to APA
%\usepackage{amsmath} % Required for some math elements 

\setlength\parindent{0pt} % Removes all indentation from paragraphs

\renewcommand{\labelenumi}{\alph{enumi}.} % Make numbering in the enumerate environment by letter rather than number (e.g. section 6)

%\usepackage{times} % Uncomment to use the Times New Roman font

%----------------------------------------------------------------------------------------
%	DOCUMENT INFORMATION
%----------------------------------------------------------------------------------------

\title{GO-terms enriched in heatmap gene clusters for mouse and \textit{Eimeria falciformis}} % Title

\author{Totta \textsc{Kasemo}} % Author name

\date{\today} % Date for the report

\begin{document}

\maketitle % Insert the title, author and date

\begin{center}
\begin{tabular}{l r}
%Date Performed: & January 1, 2012 \\ % Date the experiment was performed
%Partners: & James Smith \\ % Partner names
%& Mary Smith \\
%Instructor: & Professor Smith % Instructor/supervisor
\end{tabular}
\end{center}

% If you wish to include an abstract, uncomment the lines below
% \begin{abstract}
% Abstract text
% \end{abstract}

%----------------------------------------------------------------------------------------
%	SECTION 1
%----------------------------------------------------------------------------------------

\section{Objective}
bla bla bla
\newpage
%\begin{center}\ce{}\end{center}

%----------------------------------------------------------------------------------------
%	SECTION 2 
%----------------------------------------------------------------------------------------

\section{Results for \textit{E. falciformis}}
  Cluster 1, BP, Ef.
  Overrepresented GO biological process terms among parasite genes which are 
  highly abundant on day 7 p.i.. Compared to cluster 2, which also have a high
  abundance on day 7, these genes cluster more closely with genes in cluster 4, 
  which are highly expressed in oocysts.\\
  %\newpage
  % latex table generated in R 3.2.3 by xtable 1.8-2 package
% Thu Apr 14 14:57:16 2016
\begin{table}[ht]
\centering
\begin{tabular}{llrrrrr}
  \hline
GO.ID & Term & Annotated & Significant & Expected & p.value & adj.p \\ 
  \hline
GO:0007596 & blood coagulation &   5 &   3 & 0.18 & 0.00 & 0.01 \\ 
  GO:0007599 & hemostasis &   5 &   3 & 0.18 & 0.00 & 0.01 \\ 
  GO:0009611 & response to wounding &   5 &   3 & 0.18 & 0.00 & 0.01 \\ 
  GO:0042060 & wound healing &   5 &   3 & 0.18 & 0.00 & 0.01 \\ 
  GO:0050817 & coagulation &   5 &   3 & 0.18 & 0.00 & 0.01 \\ 
  GO:0050878 & regulation of body fluid levels &   5 &   3 & 0.18 & 0.00 & 0.01 \\ 
  GO:0032501 & multicellular organismal process &   6 &   3 & 0.22 & 0.00 & 0.01 \\ 
  GO:0044707 & single-multicellular organism process &   6 &   3 & 0.22 & 0.00 & 0.01 \\ 
  GO:0006260 & DNA replication &  34 &   6 & 1.26 & 0.00 & 0.01 \\ 
  GO:0006270 & DNA replication initiation &   3 &   2 & 0.11 & 0.00 & 0.04 \\ 
   \hline
\end{tabular}
\end{table}


  Cluster 2, BP, Ef.
  Overrepresented GO biological process terms among parasite genes which are 
  highly abundant on day 7 p.i.. Compared to cluster 1, which is also upregulated 
  on day 7 p.i., these genes cluster more closely with cluster 6, which contains
  genes highly abundant in sporozoites.\\
  % latex table generated in R 3.2.3 by xtable 1.8-2 package
% Thu Apr 14 14:57:23 2016
\begin{table}[ht]
\centering
\begin{tabular}{llrrrrr}
  \hline
GO.ID & Term & Annotated & Significant & Expected & p.value & adj.p \\ 
  \hline
GO:0006928 & movement of cell or subcellular componen... &  40 &  13 & 1.43 & 0.00 & 0.00 \\ 
  GO:0007018 & microtubule-based movement &  38 &  12 & 1.36 & 0.00 & 0.00 \\ 
  GO:0007017 & microtubule-based process &  48 &  13 & 1.71 & 0.00 & 0.00 \\ 
  GO:0044763 & single-organism cellular process & 518 &  34 & 18.50 & 0.00 & 0.00 \\ 
  GO:0044699 & single-organism process & 670 &  38 & 23.93 & 0.00 & 0.00 \\ 
  GO:0006165 & nucleoside diphosphate phosphorylation &  17 &   5 & 0.61 & 0.00 & 0.00 \\ 
  GO:0046939 & nucleotide phosphorylation &  18 &   5 & 0.64 & 0.00 & 0.00 \\ 
  GO:0009132 & nucleoside diphosphate metabolic process &  19 &   5 & 0.68 & 0.00 & 0.00 \\ 
  GO:0005975 & carbohydrate metabolic process &  67 &   9 & 2.39 & 0.00 & 0.00 \\ 
  GO:0009144 & purine nucleoside triphosphate metabolic... &  32 &   6 & 1.14 & 0.00 & 0.01 \\ 
  GO:0009199 & ribonucleoside triphosphate metabolic pr... &  32 &   6 & 1.14 & 0.00 & 0.01 \\ 
  GO:0009205 & purine ribonucleoside triphosphate metab... &  32 &   6 & 1.14 & 0.00 & 0.01 \\ 
  GO:0009141 & nucleoside triphosphate metabolic proces... &  33 &   6 & 1.18 & 0.00 & 0.01 \\ 
  GO:0006096 & glycolytic process &  14 &   4 & 0.50 & 0.00 & 0.01 \\ 
  GO:0006757 & ATP generation from ADP &  14 &   4 & 0.50 & 0.00 & 0.01 \\ 
  GO:0009135 & purine nucleoside diphosphate metabolic ... &  14 &   4 & 0.50 & 0.00 & 0.01 \\ 
  GO:0009179 & purine ribonucleoside diphosphate metabo... &  14 &   4 & 0.50 & 0.00 & 0.01 \\ 
  GO:0009185 & ribonucleoside diphosphate metabolic pro... &  14 &   4 & 0.50 & 0.00 & 0.01 \\ 
  GO:0016052 & carbohydrate catabolic process &  14 &   4 & 0.50 & 0.00 & 0.01 \\ 
  GO:0044724 & single-organism carbohydrate catabolic p... &  14 &   4 & 0.50 & 0.00 & 0.01 \\ 
  GO:0046031 & ADP metabolic process &  14 &   4 & 0.50 & 0.00 & 0.01 \\ 
  GO:0006006 & glucose metabolic process &   7 &   3 & 0.25 & 0.00 & 0.01 \\ 
  GO:0006090 & pyruvate metabolic process &  15 &   4 & 0.54 & 0.00 & 0.01 \\ 
  GO:0009150 & purine ribonucleotide metabolic process &  37 &   6 & 1.32 & 0.00 & 0.01 \\ 
  GO:0042278 & purine nucleoside metabolic process &  37 &   6 & 1.32 & 0.00 & 0.01 \\ 
  GO:0046128 & purine ribonucleoside metabolic process &  37 &   6 & 1.32 & 0.00 & 0.01 \\ 
  GO:0006163 & purine nucleotide metabolic process &  38 &   6 & 1.36 & 0.00 & 0.01 \\ 
  GO:0009259 & ribonucleotide metabolic process &  38 &   6 & 1.36 & 0.00 & 0.01 \\ 
  GO:0044723 & single-organism carbohydrate metabolic p... &  39 &   6 & 1.39 & 0.00 & 0.01 \\ 
  GO:0009119 & ribonucleoside metabolic process &  40 &   6 & 1.43 & 0.00 & 0.01 \\ 
  GO:0072521 & purine-containing compound metabolic pro... &  40 &   6 & 1.43 & 0.00 & 0.01 \\ 
  GO:0046034 & ATP metabolic process &  29 &   5 & 1.04 & 0.00 & 0.01 \\ 
  GO:0019693 & ribose phosphate metabolic process &  42 &   6 & 1.50 & 0.00 & 0.01 \\ 
  GO:0019318 & hexose metabolic process &   9 &   3 & 0.32 & 0.00 & 0.01 \\ 
  GO:0009117 & nucleotide metabolic process &  57 &   7 & 2.04 & 0.00 & 0.01 \\ 
  GO:0016310 & phosphorylation & 163 &  13 & 5.82 & 0.00 & 0.01 \\ 
  GO:0006094 & gluconeogenesis &   3 &   2 & 0.11 & 0.00 & 0.01 \\ 
  GO:0019319 & hexose biosynthetic process &   3 &   2 & 0.11 & 0.00 & 0.01 \\ 
  GO:0046364 & monosaccharide biosynthetic process &   3 &   2 & 0.11 & 0.00 & 0.01 \\ 
  GO:0005996 & monosaccharide metabolic process &  10 &   3 & 0.36 & 0.00 & 0.01 \\ 
  GO:0006753 & nucleoside phosphate metabolic process &  60 &   7 & 2.14 & 0.00 & 0.01 \\ 
  GO:0019362 & pyridine nucleotide metabolic process &  20 &   4 & 0.71 & 0.00 & 0.01 \\ 
  GO:0046496 & nicotinamide nucleotide metabolic proces... &  20 &   4 & 0.71 & 0.00 & 0.01 \\ 
  GO:0072524 & pyridine-containing compound metabolic p... &  20 &   4 & 0.71 & 0.00 & 0.01 \\ 
  GO:0009116 & nucleoside metabolic process &  46 &   6 & 1.64 & 0.00 & 0.01 \\ 
  GO:1901657 & glycosyl compound metabolic process &  46 &   6 & 1.64 & 0.00 & 0.01 \\ 
  GO:1901135 & carbohydrate derivative metabolic proces... &  78 &   8 & 2.79 & 0.01 & 0.01 \\ 
  GO:0006733 & oxidoreduction coenzyme metabolic proces... &  21 &   4 & 0.75 & 0.01 & 0.01 \\ 
  GO:0009126 & purine nucleoside monophosphate metaboli... &  34 &   5 & 1.21 & 0.01 & 0.01 \\ 
  GO:0009167 & purine ribonucleoside monophosphate meta... &  34 &   5 & 1.21 & 0.01 & 0.01 \\ 
  GO:0009161 & ribonucleoside monophosphate metabolic p... &  35 &   5 & 1.25 & 0.01 & 0.01 \\ 
  GO:0016051 & carbohydrate biosynthetic process &  12 &   3 & 0.43 & 0.01 & 0.01 \\ 
  GO:0044712 & single-organism catabolic process &  23 &   4 & 0.82 & 0.01 & 0.01 \\ 
  GO:0009123 & nucleoside monophosphate metabolic proce... &  36 &   5 & 1.29 & 0.01 & 0.01 \\ 
   \hline
\end{tabular}
\end{table}


  Cluster 1, MF, Ef.\\
  Overrepresented GO molecular function terms among parasite genes which are 
  highly abundant on day 7 p.i.. Compared to cluster 2, which also have a high
  abundance on day 7, these genes cluster more closely with genes in cluster 4, 
  which are highly expressed in oocysts.\\
  % latex table generated in R 3.2.3 by xtable 1.8-2 package
% Fri Jun  3 15:09:04 2016
\begin{table}[ht]
\centering
\begin{tabular}{llrrrrr}
  \hline
GO.ID & Term & Annotated & Significant & Expected & p.value & adj.p \\ 
  \hline
GO:0048037 & cofactor binding &  79 &   8 & 2.80 & 0.01 & 0.14 \\ 
  GO:0031177 & phosphopantetheine binding &   4 &   2 & 0.14 & 0.01 & 0.14 \\ 
  GO:0072341 & modified amino acid binding &   4 &   2 & 0.14 & 0.01 & 0.14 \\ 
  GO:0050662 & coenzyme binding &  52 &   6 & 1.84 & 0.01 & 0.14 \\ 
   \hline
\end{tabular}
\end{table}

  
  Cluster 2, MF, Ef.
  Overrepresented GO molecular function terms among parasite genes which are 
  highly abundant on day 7 p.i.. Compared to cluster 1, which is also upregulated 
  on day 7 p.i., these genes cluster more closely with cluster 6, which contains
  genes highly abundant in sporozoites.\\
  % latex table generated in R 3.2.3 by xtable 1.8-2 package
% Fri Jun  3 15:09:09 2016
\begin{table}[ht]
\centering
\begin{tabular}{llrrrrr}
  \hline
GO.ID & Term & Annotated & Significant & Expected & p.value & adj.p \\ 
  \hline
GO:0003777 & microtubule motor activity &  32 &  12 & 1.26 & 0.00 & 0.00 \\ 
  GO:0003774 & motor activity &  43 &  12 & 1.69 & 0.00 & 0.00 \\ 
  GO:0008603 & cAMP-dependent protein kinase regulator ... &   7 &   4 & 0.28 & 0.00 & 0.00 \\ 
  GO:0005509 & calcium ion binding &  72 &  11 & 2.84 & 0.00 & 0.00 \\ 
  GO:0019207 & kinase regulator activity &  13 &   4 & 0.51 & 0.00 & 0.02 \\ 
  GO:0019887 & protein kinase regulator activity &  13 &   4 & 0.51 & 0.00 & 0.02 \\ 
  GO:0004347 & glucose-6-phosphate isomerase activity &   2 &   2 & 0.08 & 0.00 & 0.02 \\ 
  GO:0005515 & protein binding & 616 &  37 & 24.26 & 0.00 & 0.03 \\ 
  GO:0008061 & chitin binding &   4 &   2 & 0.16 & 0.01 & 0.10 \\ 
   \hline
\end{tabular}
\end{table}

  
  Cluster 3, BP, Ef.
  Overrepresented GO biological process terms among parasite genes which have a low
  abundance in sporulated oocysts and sporozoites. In mouse stages, i.e., day 3, 5, and 7 samples
  there is a tendency to upregulation among these genes.\\
  % latex table generated in R 3.2.3 by xtable 1.8-2 package
% Fri Jun  3 15:09:18 2016
\begin{table}[ht]
\centering
\begin{tabular}{llrrrrr}
  \hline
GO.ID & Term & Annotated & Significant & Expected & p.value & adj.p \\ 
  \hline
GO:0044281 & small molecule metabolic process & 183 &  31 & 10.19 & 0.00 & 0.00 \\ 
  GO:1901564 & organonitrogen compound metabolic proces... & 294 &  40 & 16.37 & 0.00 & 0.00 \\ 
  GO:1901566 & organonitrogen compound biosynthetic pro... & 240 &  33 & 13.37 & 0.00 & 0.00 \\ 
  GO:0006082 & organic acid metabolic process & 109 &  21 & 6.07 & 0.00 & 0.00 \\ 
  GO:0019752 & carboxylic acid metabolic process & 109 &  21 & 6.07 & 0.00 & 0.00 \\ 
  GO:0043436 & oxoacid metabolic process & 109 &  21 & 6.07 & 0.00 & 0.00 \\ 
  GO:0044249 & cellular biosynthetic process & 426 &  46 & 23.72 & 0.00 & 0.00 \\ 
  GO:0009058 & biosynthetic process & 446 &  47 & 24.84 & 0.00 & 0.00 \\ 
  GO:0044710 & single-organism metabolic process & 379 &  42 & 21.11 & 0.00 & 0.00 \\ 
  GO:1901576 & organic substance biosynthetic process & 428 &  45 & 23.84 & 0.00 & 0.00 \\ 
  GO:0044711 & single-organism biosynthetic process & 103 &  18 & 5.74 & 0.00 & 0.00 \\ 
  GO:0009126 & purine nucleoside monophosphate metaboli... &  34 &  10 & 1.89 & 0.00 & 0.00 \\ 
  GO:0009167 & purine ribonucleoside monophosphate meta... &  34 &  10 & 1.89 & 0.00 & 0.00 \\ 
  GO:0032787 & monocarboxylic acid metabolic process &  28 &   9 & 1.56 & 0.00 & 0.00 \\ 
  GO:0009161 & ribonucleoside monophosphate metabolic p... &  35 &  10 & 1.95 & 0.00 & 0.00 \\ 
  GO:0006753 & nucleoside phosphate metabolic process &  60 &  13 & 3.34 & 0.00 & 0.00 \\ 
  GO:0009123 & nucleoside monophosphate metabolic proce... &  36 &  10 & 2.00 & 0.00 & 0.00 \\ 
  GO:0009150 & purine ribonucleotide metabolic process &  37 &  10 & 2.06 & 0.00 & 0.00 \\ 
  GO:0044271 & cellular nitrogen compound biosynthetic ... & 339 &  36 & 18.88 & 0.00 & 0.00 \\ 
  GO:0055086 & nucleobase-containing small molecule met... &  72 &  14 & 4.01 & 0.00 & 0.00 \\ 
  GO:0006163 & purine nucleotide metabolic process &  38 &  10 & 2.12 & 0.00 & 0.00 \\ 
  GO:0009259 & ribonucleotide metabolic process &  38 &  10 & 2.12 & 0.00 & 0.00 \\ 
  GO:0009144 & purine nucleoside triphosphate metabolic... &  32 &   9 & 1.78 & 0.00 & 0.00 \\ 
  GO:0009199 & ribonucleoside triphosphate metabolic pr... &  32 &   9 & 1.78 & 0.00 & 0.00 \\ 
  GO:0009205 & purine ribonucleoside triphosphate metab... &  32 &   9 & 1.78 & 0.00 & 0.00 \\ 
  GO:0009119 & ribonucleoside metabolic process &  40 &  10 & 2.23 & 0.00 & 0.00 \\ 
  GO:0072521 & purine-containing compound metabolic pro... &  40 &  10 & 2.23 & 0.00 & 0.00 \\ 
  GO:0009117 & nucleotide metabolic process &  57 &  12 & 3.17 & 0.00 & 0.00 \\ 
  GO:0009132 & nucleoside diphosphate metabolic process &  19 &   7 & 1.06 & 0.00 & 0.00 \\ 
  GO:0009141 & nucleoside triphosphate metabolic proces... &  33 &   9 & 1.84 & 0.00 & 0.00 \\ 
  GO:0006807 & nitrogen compound metabolic process & 591 &  51 & 32.91 & 0.00 & 0.00 \\ 
  GO:0019693 & ribose phosphate metabolic process &  42 &  10 & 2.34 & 0.00 & 0.00 \\ 
  GO:0008152 & metabolic process & 1280 &  85 & 71.28 & 0.00 & 0.00 \\ 
  GO:0046034 & ATP metabolic process &  29 &   8 & 1.62 & 0.00 & 0.00 \\ 
  GO:0042278 & purine nucleoside metabolic process &  37 &   9 & 2.06 & 0.00 & 0.00 \\ 
  GO:0046128 & purine ribonucleoside metabolic process &  37 &   9 & 2.06 & 0.00 & 0.00 \\ 
  GO:0009116 & nucleoside metabolic process &  46 &  10 & 2.56 & 0.00 & 0.00 \\ 
  GO:1901657 & glycosyl compound metabolic process &  46 &  10 & 2.56 & 0.00 & 0.00 \\ 
  GO:0072330 & monocarboxylic acid biosynthetic process &  11 &   5 & 0.61 & 0.00 & 0.00 \\ 
  GO:0034641 & cellular nitrogen compound metabolic pro... & 563 &  48 & 31.35 & 0.00 & 0.00 \\ 
  GO:0006165 & nucleoside diphosphate phosphorylation &  17 &   6 & 0.95 & 0.00 & 0.00 \\ 
  GO:0019637 & organophosphate metabolic process &  88 &  14 & 4.90 & 0.00 & 0.00 \\ 
  GO:1901293 & nucleoside phosphate biosynthetic proces... &  32 &   8 & 1.78 & 0.00 & 0.00 \\ 
  GO:0046939 & nucleotide phosphorylation &  18 &   6 & 1.00 & 0.00 & 0.00 \\ 
  GO:0043604 & amide biosynthetic process & 185 &  22 & 10.30 & 0.00 & 0.00 \\ 
  GO:0006412 & translation & 175 &  21 & 9.75 & 0.00 & 0.00 \\ 
  GO:0043603 & cellular amide metabolic process & 189 &  22 & 10.53 & 0.00 & 0.00 \\ 
  GO:0043043 & peptide biosynthetic process & 177 &  21 & 9.86 & 0.00 & 0.00 \\ 
  GO:0006091 & generation of precursor metabolites and ... &  27 &   7 & 1.50 & 0.00 & 0.00 \\ 
  GO:0006633 & fatty acid biosynthetic process &   8 &   4 & 0.45 & 0.00 & 0.00 \\ 
  GO:0006518 & peptide metabolic process & 181 &  21 & 10.08 & 0.00 & 0.00 \\ 
  GO:0016053 & organic acid biosynthetic process &  28 &   7 & 1.56 & 0.00 & 0.00 \\ 
  GO:0046394 & carboxylic acid biosynthetic process &  28 &   7 & 1.56 & 0.00 & 0.00 \\ 
  GO:0006096 & glycolytic process &  14 &   5 & 0.78 & 0.00 & 0.00 \\ 
  GO:0006757 & ATP generation from ADP &  14 &   5 & 0.78 & 0.00 & 0.00 \\ 
  GO:0009135 & purine nucleoside diphosphate metabolic ... &  14 &   5 & 0.78 & 0.00 & 0.00 \\ 
  GO:0009179 & purine ribonucleoside diphosphate metabo... &  14 &   5 & 0.78 & 0.00 & 0.00 \\ 
  GO:0009185 & ribonucleoside diphosphate metabolic pro... &  14 &   5 & 0.78 & 0.00 & 0.00 \\ 
  GO:0016052 & carbohydrate catabolic process &  14 &   5 & 0.78 & 0.00 & 0.00 \\ 
  GO:0044724 & single-organism carbohydrate catabolic p... &  14 &   5 & 0.78 & 0.00 & 0.00 \\ 
  GO:0046031 & ADP metabolic process &  14 &   5 & 0.78 & 0.00 & 0.00 \\ 
  GO:1901135 & carbohydrate derivative metabolic proces... &  78 &  12 & 4.34 & 0.00 & 0.00 \\ 
  GO:0006090 & pyruvate metabolic process &  15 &   5 & 0.84 & 0.00 & 0.00 \\ 
  GO:0009127 & purine nucleoside monophosphate biosynth... &  15 &   5 & 0.84 & 0.00 & 0.00 \\ 
  GO:0009168 & purine ribonucleoside monophosphate bios... &  15 &   5 & 0.84 & 0.00 & 0.00 \\ 
  GO:0009165 & nucleotide biosynthetic process &  30 &   7 & 1.67 & 0.00 & 0.00 \\ 
  GO:0044712 & single-organism catabolic process &  23 &   6 & 1.28 & 0.00 & 0.00 \\ 
  GO:0009156 & ribonucleoside monophosphate biosyntheti... &  16 &   5 & 0.89 & 0.00 & 0.00 \\ 
  GO:0044237 & cellular metabolic process & 956 &  67 & 53.24 & 0.00 & 0.00 \\ 
  GO:0044283 & small molecule biosynthetic process &  33 &   7 & 1.84 & 0.00 & 0.00 \\ 
  GO:0009124 & nucleoside monophosphate biosynthetic pr... &  17 &   5 & 0.95 & 0.00 & 0.00 \\ 
  GO:0006631 & fatty acid metabolic process &  11 &   4 & 0.61 & 0.00 & 0.00 \\ 
  GO:0009152 & purine ribonucleotide biosynthetic proce... &  18 &   5 & 1.00 & 0.00 & 0.00 \\ 
  GO:0006164 & purine nucleotide biosynthetic process &  19 &   5 & 1.06 & 0.00 & 0.00 \\ 
  GO:0009260 & ribonucleotide biosynthetic process &  19 &   5 & 1.06 & 0.00 & 0.00 \\ 
  GO:0046390 & ribose phosphate biosynthetic process &  19 &   5 & 1.06 & 0.00 & 0.00 \\ 
  GO:0009266 & response to temperature stimulus &   2 &   2 & 0.11 & 0.00 & 0.00 \\ 
  GO:1901137 & carbohydrate derivative biosynthetic pro... &  37 &   7 & 2.06 & 0.00 & 0.00 \\ 
  GO:0008610 & lipid biosynthetic process &  28 &   6 & 1.56 & 0.00 & 0.00 \\ 
  GO:0090407 & organophosphate biosynthetic process &  47 &   8 & 2.62 & 0.00 & 0.00 \\ 
  GO:0019362 & pyridine nucleotide metabolic process &  20 &   5 & 1.11 & 0.00 & 0.00 \\ 
  GO:0046496 & nicotinamide nucleotide metabolic proces... &  20 &   5 & 1.11 & 0.00 & 0.00 \\ 
  GO:0072524 & pyridine-containing compound metabolic p... &  20 &   5 & 1.11 & 0.00 & 0.00 \\ 
  GO:0051186 & cofactor metabolic process &  38 &   7 & 2.12 & 0.00 & 0.00 \\ 
  GO:0009142 & nucleoside triphosphate biosynthetic pro... &  13 &   4 & 0.72 & 0.00 & 0.01 \\ 
  GO:0009145 & purine nucleoside triphosphate biosynthe... &  13 &   4 & 0.72 & 0.00 & 0.01 \\ 
  GO:0009201 & ribonucleoside triphosphate biosynthetic... &  13 &   4 & 0.72 & 0.00 & 0.01 \\ 
  GO:0009206 & purine ribonucleoside triphosphate biosy... &  13 &   4 & 0.72 & 0.00 & 0.01 \\ 
  GO:0006733 & oxidoreduction coenzyme metabolic proces... &  21 &   5 & 1.17 & 0.00 & 0.01 \\ 
  GO:0009163 & nucleoside biosynthetic process &  21 &   5 & 1.17 & 0.00 & 0.01 \\ 
  GO:0042455 & ribonucleoside biosynthetic process &  21 &   5 & 1.17 & 0.00 & 0.01 \\ 
  GO:0072522 & purine-containing compound biosynthetic ... &  21 &   5 & 1.17 & 0.00 & 0.01 \\ 
  GO:1901659 & glycosyl compound biosynthetic process &  21 &   5 & 1.17 & 0.00 & 0.01 \\ 
  GO:0006221 & pyrimidine nucleotide biosynthetic proce... &   7 &   3 & 0.39 & 0.00 & 0.01 \\ 
  GO:0015985 & energy coupled proton transport, down el... &   7 &   3 & 0.39 & 0.00 & 0.01 \\ 
  GO:0015986 & ATP synthesis coupled proton transport &   7 &   3 & 0.39 & 0.00 & 0.01 \\ 
  GO:0071704 & organic substance metabolic process & 1090 &  72 & 60.70 & 0.01 & 0.01 \\ 
  GO:0010467 & gene expression & 353 &  30 & 19.66 & 0.01 & 0.01 \\ 
  GO:0006220 & pyrimidine nucleotide metabolic process &   8 &   3 & 0.45 & 0.01 & 0.01 \\ 
  GO:0006414 & translational elongation &  15 &   4 & 0.84 & 0.01 & 0.01 \\ 
   \hline
\end{tabular}
\end{table}

  
  Cluster 3, MF, Ef.
  Overrepresented GO molecular function terms among parasite genes which have a low
  abundance in sporulated oocysts and sporozoites. In mouse stages, i.e., day 3, 5, and 7 samples
  there is a tendency to upregulation among these genes.\\
  % latex table generated in R 3.2.3 by xtable 1.8-2 package
% Fri Jun  3 15:09:14 2016
\begin{table}[ht]
\centering
\begin{tabular}{llrrrrr}
  \hline
GO.ID & Term & Annotated & Significant & Expected & p.value & adj.p \\ 
  \hline
GO:0004312 & fatty acid synthase activity &   2 &   2 & 0.09 & 0.00 & 0.07 \\ 
  GO:0051920 & peroxiredoxin activity &   2 &   2 & 0.09 & 0.00 & 0.07 \\ 
  GO:0003824 & catalytic activity & 1318 &  74 & 58.58 & 0.00 & 0.07 \\ 
  GO:0016746 & transferase activity, transferring acyl ... &  46 &   7 & 2.04 & 0.00 & 0.08 \\ 
  GO:0019205 & nucleobase-containing compound kinase ac... &   8 &   3 & 0.36 & 0.00 & 0.08 \\ 
  GO:0016615 & malate dehydrogenase activity &   3 &   2 & 0.13 & 0.01 & 0.09 \\ 
  GO:0003746 & translation elongation factor activity &  10 &   3 & 0.44 & 0.01 & 0.09 \\ 
  GO:0008135 & translation factor activity, RNA binding &  30 &   5 & 1.33 & 0.01 & 0.09 \\ 
   \hline
\end{tabular}
\end{table}

  
  Cluster 4, BP, Ef.
  Overrepresented GO biological process terms among parasite genes which are highly abundant in
  sporulated oocysts. These genes cluster most closely with cluster 1, in which genes are highly
  abundant on day 7 p.i.. In cluster 4, genes have below average abundance in all day 3 and 5 samples.\\
  % latex table generated in R 3.2.3 by xtable 1.8-2 package
% Fri Jun  3 15:09:23 2016
\begin{table}[ht]
\centering
\begin{tabular}{llrrrrr}
  \hline
GO.ID & Term & Annotated & Significant & Expected & p.value & adj.p \\ 
  \hline
GO:0030154 & cell differentiation &   3 &   3 & 0.17 & 0.00 & 0.02 \\ 
  GO:0032502 & developmental process &   4 &   3 & 0.23 & 0.00 & 0.02 \\ 
  GO:0044767 & single-organism developmental process &   4 &   3 & 0.23 & 0.00 & 0.02 \\ 
  GO:0048869 & cellular developmental process &   4 &   3 & 0.23 & 0.00 & 0.02 \\ 
  GO:0009062 & fatty acid catabolic process &   2 &   2 & 0.12 & 0.00 & 0.04 \\ 
  GO:0016042 & lipid catabolic process &   2 &   2 & 0.12 & 0.00 & 0.04 \\ 
  GO:0044242 & cellular lipid catabolic process &   2 &   2 & 0.12 & 0.00 & 0.04 \\ 
  GO:0072329 & monocarboxylic acid catabolic process &   2 &   2 & 0.12 & 0.00 & 0.04 \\ 
  GO:0006563 & L-serine metabolic process &   3 &   2 & 0.17 & 0.01 & 0.10 \\ 
   \hline
\end{tabular}
\end{table}

  
  Cluster 4, MF, Ef.
  Overrepresented GO molecular function terms among parasite genes which are highly abundant in
  sporulated oocysts. These genes cluster most closely with cluster 1, in which genes are highly
  abundant on day 7 p.i.. In cluster 4, genes have below average abundance in all day 3 and 5 samples.\\
  % latex table generated in R 3.2.3 by xtable 1.8-2 package
% Thu Apr 14 14:57:33 2016
\begin{table}[ht]
\centering
\begin{tabular}{llrrrrr}
  \hline
GO.ID & Term & Annotated & Significant & Expected & p.value & adj.p \\ 
  \hline
GO:0004871 & signal transducer activity &   7 &   4 & 0.39 & 0.00 & 0.03 \\ 
  GO:0005057 & receptor signaling protein activity &   4 &   3 & 0.22 & 0.00 & 0.03 \\ 
  GO:0060089 & molecular transducer activity &   9 &   4 & 0.50 & 0.00 & 0.03 \\ 
  GO:0048037 & cofactor binding &  79 &  12 & 4.37 & 0.00 & 0.03 \\ 
  GO:0005496 & steroid binding &   2 &   2 & 0.11 & 0.00 & 0.04 \\ 
  GO:0032934 & sterol binding &   2 &   2 & 0.11 & 0.00 & 0.04 \\ 
  GO:0043178 & alcohol binding &   2 &   2 & 0.11 & 0.00 & 0.04 \\ 
  GO:0008233 & peptidase activity & 113 &  14 & 6.26 & 0.00 & 0.04 \\ 
  GO:0016614 & oxidoreductase activity, acting on CH-OH... &  21 &   5 & 1.16 & 0.00 & 0.05 \\ 
  GO:0070011 & peptidase activity, acting on L-amino ac... & 107 &  13 & 5.92 & 0.01 & 0.05 \\ 
  GO:0004252 & serine-type endopeptidase activity &  22 &   5 & 1.22 & 0.01 & 0.05 \\ 
  GO:0030170 & pyridoxal phosphate binding &  22 &   5 & 1.22 & 0.01 & 0.05 \\ 
  GO:0004702 & receptor signaling protein serine/threon... &   3 &   2 & 0.17 & 0.01 & 0.06 \\ 
  GO:0004707 & MAP kinase activity &   3 &   2 & 0.17 & 0.01 & 0.06 \\ 
  GO:0005102 & receptor binding &   3 &   2 & 0.17 & 0.01 & 0.06 \\ 
   \hline
\end{tabular}
\end{table}

  
  Cluster 5, BP, Ef.
  Overrepresented GO biological process terms among parasite genes which have low abundance
  in sporulated oocysts, sporozoites and on day 7 p.i.. These genes seem to increase in abundance
  upon infection and decrease late in infection, i.e., on day 7.\\
  \input{../cluster5_BP_Ef_OoSpoDown_day5Up.tex}
  
  Cluster 5, MF, Ef.
  Overrepresented GO molecular function terms among parasite genes which have low abundance
  in sporulated oocysts, sporozoites and on day 7 p.i.. These genes seem to increase in abundance
  upon infection and decrease late in infection, i.e., on day 7.\\
  % latex table generated in R 3.2.3 by xtable 1.8-2 package
% Fri Jun  3 15:09:25 2016
\begin{table}[ht]
\centering
\begin{tabular}{llrrrrr}
  \hline
GO.ID & Term & Annotated & Significant & Expected & p.value & adj.p \\ 
  \hline
GO:0030145 & manganese ion binding &   4 &   2 & 0.12 & 0.01 & 0.12 \\ 
   \hline
\end{tabular}
\end{table}

  
  Cluster 6, BP, Ef.
  Overrepresented GO biological process terms among parasite genes which are highly abundant in sporozoites
  but mainly downregulated in sporulated oocysts, apart from a few genes in the cluster.
  This cluster seems to distinguish sporozoites from sporulated oocysts and have an average mRNA abundance in
  all other samples.\\
  % latex table generated in R 3.2.3 by xtable 1.8-2 package
% Thu Apr 14 14:57:52 2016
\begin{table}[ht]
\centering
\begin{tabular}{llrrrrr}
  \hline
GO.ID & Term & Annotated & Significant & Expected & p.value & adj.p \\ 
  \hline
GO:0007154 & cell communication &  63 &   8 & 1.68 & 0.00 & 0.02 \\ 
  GO:0007165 & signal transduction &  58 &   7 & 1.54 & 0.00 & 0.02 \\ 
  GO:0023052 & signaling &  58 &   7 & 1.54 & 0.00 & 0.02 \\ 
  GO:0044700 & single organism signaling &  58 &   7 & 1.54 & 0.00 & 0.02 \\ 
  GO:0007186 & G-protein coupled receptor signaling pat... &   3 &   2 & 0.08 & 0.00 & 0.03 \\ 
  GO:0007205 & protein kinase C-activating G-protein co... &   3 &   2 & 0.08 & 0.00 & 0.03 \\ 
  GO:0051716 & cellular response to stimulus &  99 &   8 & 2.64 & 0.00 & 0.05 \\ 
  GO:0009405 & pathogenesis &   5 &   2 & 0.13 & 0.01 & 0.07 \\ 
  GO:0051704 & multi-organism process &   5 &   2 & 0.13 & 0.01 & 0.07 \\ 
   \hline
\end{tabular}
\end{table}

  
  Cluster 6, MF, Ef.
  Overrepresented GO molecular function terms among parasite genes which are highly abundant in sporozoites
  but mainly downregulated in sporulated oocysts, apart from a few genes in the cluster.
  This cluster seems to distinguish sporozoites from sporulated oocysts and have an average mRNA abundance in
  all other samples.\\
  % latex table generated in R 3.2.3 by xtable 1.8-2 package
% Thu Apr 14 14:57:47 2016
\begin{table}[ht]
\centering
\begin{tabular}{llrrrrr}
  \hline
GO.ID & Term & Annotated & Significant & Expected & p.value & adj.p \\ 
  \hline
GO:0004143 & diacylglycerol kinase activity &   3 &   2 & 0.07 & 0.00 & 0.17 \\ 
  GO:0005543 & phospholipid binding &  13 &   3 & 0.32 & 0.00 & 0.17 \\ 
  GO:0008289 & lipid binding &  16 &   3 & 0.39 & 0.01 & 0.19 \\ 
  GO:0003951 & NAD+ kinase activity &   6 &   2 & 0.15 & 0.01 & 0.19 \\ 
   \hline
\end{tabular}
\end{table}

  
  Cluster 7, BP, Ef.
  Overrepresented GO biological process terms among parasite genes which have low abundance in sporulated oocysts.
  In sporozoites and early in infection, i.e., day 3 and 5 p.i., some of these genes are highly abundant. On day
  7 p.i. these mRNAs have a below average abundance.\\
  % latex table generated in R 3.2.3 by xtable 1.8-2 package
% Thu Apr 14 14:57:59 2016
\begin{table}[ht]
\centering
\begin{tabular}{llrrrrr}
  \hline
GO.ID & Term & Annotated & Significant & Expected & p.value & adj.p \\ 
  \hline
GO:0034660 & ncRNA metabolic process &  79 &  13 & 3.11 & 0.00 & 0.00 \\ 
  GO:0006399 & tRNA metabolic process &  62 &  10 & 2.44 & 0.00 & 0.00 \\ 
  GO:0006418 & tRNA aminoacylation for protein translat... &  40 &   7 & 1.57 & 0.00 & 0.02 \\ 
  GO:0043038 & amino acid activation &  42 &   7 & 1.65 & 0.00 & 0.02 \\ 
  GO:0043039 & tRNA aminoacylation &  42 &   7 & 1.65 & 0.00 & 0.02 \\ 
  GO:0016070 & RNA metabolic process & 239 &  19 & 9.40 & 0.00 & 0.02 \\ 
  GO:0046488 & phosphatidylinositol metabolic process &  15 &   4 & 0.59 & 0.00 & 0.03 \\ 
  GO:0006650 & glycerophospholipid metabolic process &  17 &   4 & 0.67 & 0.00 & 0.04 \\ 
  GO:0046486 & glycerolipid metabolic process &  17 &   4 & 0.67 & 0.00 & 0.04 \\ 
  GO:0090304 & nucleic acid metabolic process & 325 &  22 & 12.79 & 0.00 & 0.04 \\ 
  GO:0006139 & nucleobase-containing compound metabolic... & 406 &  25 & 15.97 & 0.01 & 0.07 \\ 
   \hline
\end{tabular}
\end{table}

  
  Cluster 7, MF, Ef.
  Overrepresented GO molecular function terms among parasite genes which have low abundance in sporulated oocysts.
  In sporozoites and early in infection, i.e., day 3 and 5 p.i., some of these genes are highly abundant. On day
  7 p.i. these mRNAs have a below average abundance.\\
  % latex table generated in R 3.2.3 by xtable 1.8-2 package
% Thu Apr 14 14:57:55 2016
\begin{table}[ht]
\centering
\begin{tabular}{llrrrrr}
  \hline
GO.ID & Term & Annotated & Significant & Expected & p.value & adj.p \\ 
  \hline
GO:0052689 & carboxylic ester hydrolase activity &  12 &   5 & 0.56 & 0.00 & 0.01 \\ 
  GO:0016307 & phosphatidylinositol phosphate kinase ac... &   4 &   3 & 0.19 & 0.00 & 0.02 \\ 
  GO:0004725 & protein tyrosine phosphatase activity &   6 &   3 & 0.28 & 0.00 & 0.05 \\ 
  GO:0004812 & aminoacyl-tRNA ligase activity &  41 &   7 & 1.92 & 0.00 & 0.05 \\ 
  GO:0016875 & ligase activity, forming carbon-oxygen b... &  42 &   7 & 1.96 & 0.00 & 0.05 \\ 
  GO:0016876 & ligase activity, forming aminoacyl-tRNA ... &  42 &   7 & 1.96 & 0.00 & 0.05 \\ 
  GO:0008026 & ATP-dependent helicase activity &  60 &   8 & 2.81 & 0.01 & 0.08 \\ 
  GO:0070035 & purine NTP-dependent helicase activity &  60 &   8 & 2.81 & 0.01 & 0.08 \\ 
   \hline
\end{tabular}
\end{table}


\newpage

\section{Results for mouse}
  Cluster 1, BP, Mm
  Overrepresented GO biological process terms among mouse genes which have low abundance pre infection and 
  become more abundant in most day 3 and 5 samples from immunecompetent mice, but less in immune deficient Rag1-/- mice. 
  The upregulation trend is most clear in three first infection day 5 samples, which also form a separate
  sample cluster. On day 7, most of these genes are not differentially abundant, or below average, i.e., they
  seem to peak on day 5 p.i..\\
  % latex table generated in R 3.2.3 by xtable 1.8-2 package
% Thu Mar 24 06:00:18 2016
\begin{table}[ht]
\centering
\begin{tabular}{rllrrrrr}
  \hline
 & GO.ID & Term & Annotated & Significant & Expected & p.value & adj.p \\ 
  \hline
1 & GO:0045087 & innate immune response & 340 &  61 & 7.54 & 0.00 & 0.00 \\ 
  2 & GO:0006955 & immune response & 674 &  79 & 14.95 & 0.00 & 0.00 \\ 
  3 & GO:0006952 & defense response & 724 &  75 & 16.06 & 0.00 & 0.00 \\ 
  4 & GO:0002376 & immune system process & 1326 &  98 & 29.41 & 0.00 & 0.00 \\ 
  5 & GO:0098542 & defense response to other organism & 282 &  48 & 6.25 & 0.00 & 0.00 \\ 
  6 & GO:0043207 & response to external biotic stimulus & 438 &  55 & 9.71 & 0.00 & 0.00 \\ 
  7 & GO:0051707 & response to other organism & 438 &  55 & 9.71 & 0.00 & 0.00 \\ 
  8 & GO:0009607 & response to biotic stimulus & 460 &  55 & 10.20 & 0.00 & 0.00 \\ 
  9 & GO:0002252 & immune effector process & 440 &  48 & 9.76 & 0.00 & 0.00 \\ 
  10 & GO:0009615 & response to virus & 205 &  34 & 4.55 & 0.00 & 0.00 \\ 
  11 & GO:0051607 & defense response to virus & 181 &  32 & 4.01 & 0.00 & 0.00 \\ 
  12 & GO:0034097 & response to cytokine & 346 &  42 & 7.67 & 0.00 & 0.00 \\ 
  13 & GO:0034341 & response to interferon-gamma &  47 &  18 & 1.04 & 0.00 & 0.00 \\ 
  14 & GO:0006950 & response to stress & 2042 &  98 & 45.29 & 0.00 & 0.00 \\ 
  15 & GO:0035456 & response to interferon-beta &  33 &  14 & 0.73 & 0.00 & 0.00 \\ 
  16 & GO:0071345 & cellular response to cytokine stimulus & 276 &  32 & 6.12 & 0.00 & 0.00 \\ 
  17 & GO:0045088 & regulation of innate immune response & 146 &  24 & 3.24 & 0.00 & 0.00 \\ 
  18 & GO:0051704 & multi-organism process & 959 &  61 & 21.27 & 0.00 & 0.00 \\ 
  19 & GO:0009605 & response to external stimulus & 1135 &  67 & 25.17 & 0.00 & 0.00 \\ 
  20 & GO:0002682 & regulation of immune system process & 732 &  52 & 16.23 & 0.00 & 0.00 \\ 
  21 & GO:0048525 & negative regulation of viral process &  66 &  17 & 1.46 & 0.00 & 0.00 \\ 
  22 & GO:0019882 & antigen processing and presentation &  69 &  17 & 1.53 & 0.00 & 0.00 \\ 
  23 & GO:0035458 & cellular response to interferon-beta &  27 &  12 & 0.60 & 0.00 & 0.00 \\ 
  24 & GO:1903901 & negative regulation of viral life cycle &  63 &  16 & 1.40 & 0.00 & 0.00 \\ 
  25 & GO:0043901 & negative regulation of multi-organism pr... & 103 &  19 & 2.28 & 0.00 & 0.00 \\ 
  26 & GO:0050776 & regulation of immune response & 366 &  34 & 8.12 & 0.00 & 0.00 \\ 
  27 & GO:0050792 & regulation of viral process & 118 &  20 & 2.62 & 0.00 & 0.00 \\ 
  28 & GO:0043900 & regulation of multi-organism process & 271 &  29 & 6.01 & 0.00 & 0.00 \\ 
  29 & GO:0071346 & cellular response to interferon-gamma &  32 &  12 & 0.71 & 0.00 & 0.00 \\ 
  30 & GO:0019884 & antigen processing and presentation of e... &  25 &  11 & 0.55 & 0.00 & 0.00 \\ 
  31 & GO:0044764 & multi-organism cellular process & 201 &  25 & 4.46 & 0.00 & 0.00 \\ 
  32 & GO:0043903 & regulation of symbiosis, encompassing mu... & 137 &  21 & 3.04 & 0.00 & 0.00 \\ 
  33 & GO:0045071 & negative regulation of viral genome repl... &  33 &  12 & 0.73 & 0.00 & 0.00 \\ 
  34 & GO:0044403 & symbiosis, encompassing mutualism throug... & 225 &  26 & 4.99 & 0.00 & 0.00 \\ 
  35 & GO:0044419 & interspecies interaction between organis... & 225 &  26 & 4.99 & 0.00 & 0.00 \\ 
  36 & GO:0031347 & regulation of defense response & 352 &  32 & 7.81 & 0.00 & 0.00 \\ 
  37 & GO:0048002 & antigen processing and presentation of p... &  46 &  13 & 1.02 & 0.00 & 0.00 \\ 
  38 & GO:0009617 & response to bacterium & 240 &  26 & 5.32 & 0.00 & 0.00 \\ 
  39 & GO:0045089 & positive regulation of innate immune res... & 120 &  18 & 2.66 & 0.00 & 0.00 \\ 
  40 & GO:0002478 & antigen processing and presentation of e... &  20 &   9 & 0.44 & 0.00 & 0.00 \\ 
  41 & GO:0050778 & positive regulation of immune response & 285 &  27 & 6.32 & 0.00 & 0.00 \\ 
  42 & GO:1903900 & regulation of viral life cycle & 109 &  17 & 2.42 & 0.00 & 0.00 \\ 
  43 & GO:0034340 & response to type I interferon &  22 &   9 & 0.49 & 0.00 & 0.00 \\ 
  44 & GO:0042742 & defense response to bacterium & 102 &  16 & 2.26 & 0.00 & 0.00 \\ 
  45 & GO:0002684 & positive regulation of immune system pro... & 471 &  34 & 10.45 & 0.00 & 0.00 \\ 
  46 & GO:0045069 & regulation of viral genome replication &  52 &  12 & 1.15 & 0.00 & 0.00 \\ 
  47 & GO:0031349 & positive regulation of defense response & 170 &  20 & 3.77 & 0.00 & 0.00 \\ 
  48 & GO:0019058 & viral life cycle & 153 &  19 & 3.39 & 0.00 & 0.00 \\ 
  49 & GO:0002221 & pattern recognition receptor signaling p... &  77 &  14 & 1.71 & 0.00 & 0.00 \\ 
  50 & GO:0016032 & viral process & 191 &  21 & 4.24 & 0.00 & 0.00 \\ 
  51 & GO:0002758 & innate immune response-activating signal... &  80 &  14 & 1.77 & 0.00 & 0.00 \\ 
  52 & GO:0002683 & negative regulation of immune system pro... & 257 &  24 & 5.70 & 0.00 & 0.00 \\ 
  53 & GO:0042832 & defense response to protozoan &  20 &   8 & 0.44 & 0.00 & 0.00 \\ 
  54 & GO:0002218 & activation of innate immune response &  86 &  14 & 1.91 & 0.00 & 0.00 \\ 
  55 & GO:0002474 & antigen processing and presentation of p... &  28 &   9 & 0.62 & 0.00 & 0.00 \\ 
  56 & GO:0019079 & viral genome replication &  60 &  12 & 1.33 & 0.00 & 0.00 \\ 
  57 & GO:0050896 & response to stimulus & 4167 & 136 & 92.42 & 0.00 & 0.00 \\ 
  58 & GO:0002483 & antigen processing and presentation of e... &   9 &   6 & 0.20 & 0.00 & 0.00 \\ 
  59 & GO:0010033 & response to organic substance & 1314 &  61 & 29.14 & 0.00 & 0.00 \\ 
  60 & GO:0001562 & response to protozoan &  22 &   8 & 0.49 & 0.00 & 0.00 \\ 
  61 & GO:0098586 & cellular response to virus &  22 &   8 & 0.49 & 0.00 & 0.00 \\ 
  62 & GO:0019883 & antigen processing and presentation of e... &  10 &   6 & 0.22 & 0.00 & 0.00 \\ 
  63 & GO:0001816 & cytokine production & 381 &  28 & 8.45 & 0.00 & 0.00 \\ 
  64 & GO:0035455 & response to interferon-alpha &  16 &   7 & 0.35 & 0.00 & 0.00 \\ 
  65 & GO:0001817 & regulation of cytokine production & 334 &  26 & 7.41 & 0.00 & 0.00 \\ 
  66 & GO:0002237 & response to molecule of bacterial origin & 154 &  17 & 3.42 & 0.00 & 0.00 \\ 
  67 & GO:0060337 & type I interferon signaling pathway &  18 &   7 & 0.40 & 0.00 & 0.00 \\ 
  68 & GO:0071357 & cellular response to type I interferon &  18 &   7 & 0.40 & 0.00 & 0.00 \\ 
  69 & GO:0051250 & negative regulation of lymphocyte activa... &  89 &  13 & 1.97 & 0.00 & 0.00 \\ 
  70 & GO:0080134 & regulation of response to stress & 790 &  42 & 17.52 & 0.00 & 0.00 \\ 
  71 & GO:0019885 & antigen processing and presentation of e... &   7 &   5 & 0.16 & 0.00 & 0.00 \\ 
  72 & GO:0039528 & cytoplasmic pattern recognition receptor... &  13 &   6 & 0.29 & 0.00 & 0.00 \\ 
  73 & GO:0002695 & negative regulation of leukocyte activat... &  99 &  13 & 2.20 & 0.00 & 0.00 \\ 
  74 & GO:0002753 & cytoplasmic pattern recognition receptor... &  22 &   7 & 0.49 & 0.00 & 0.00 \\ 
  75 & GO:0001819 & positive regulation of cytokine producti... & 222 &  19 & 4.92 & 0.00 & 0.00 \\ 
  76 & GO:0032606 & type I interferon production &  45 &   9 & 1.00 & 0.00 & 0.00 \\ 
  77 & GO:0002764 & immune response-regulating signaling pat... & 181 &  17 & 4.01 & 0.00 & 0.00 \\ 
  78 & GO:0048584 & positive regulation of response to stimu... & 1076 &  49 & 23.86 & 0.00 & 0.00 \\ 
  79 & GO:0050866 & negative regulation of cell activation & 108 &  13 & 2.40 & 0.00 & 0.00 \\ 
  80 & GO:0002821 & positive regulation of adaptive immune r... &  61 &  10 & 1.35 & 0.00 & 0.00 \\ 
  81 & GO:0002697 & regulation of immune effector process & 258 &  20 & 5.72 & 0.00 & 0.00 \\ 
  82 & GO:0045341 & MHC class I biosynthetic process &   5 &   4 & 0.11 & 0.00 & 0.00 \\ 
  83 & GO:0045343 & regulation of MHC class I biosynthetic p... &   5 &   4 & 0.11 & 0.00 & 0.00 \\ 
  84 & GO:0002757 & immune response-activating signal transd... & 171 &  16 & 3.79 & 0.00 & 0.00 \\ 
  85 & GO:0002699 & positive regulation of immune effector p... & 115 &  13 & 2.55 & 0.00 & 0.00 \\ 
  86 & GO:0019221 & cytokine-mediated signaling pathway & 175 &  16 & 3.88 & 0.00 & 0.00 \\ 
  87 & GO:0050868 & negative regulation of T cell activation &  66 &  10 & 1.46 & 0.00 & 0.00 \\ 
  88 & GO:0071310 & cellular response to organic substance & 981 &  45 & 21.76 & 0.00 & 0.00 \\ 
  89 & GO:0044406 & adhesion of symbiont to host &  11 &   5 & 0.24 & 0.00 & 0.00 \\ 
  90 & GO:0051249 & regulation of lymphocyte activation & 246 &  19 & 5.46 & 0.00 & 0.00 \\ 
  91 & GO:0032496 & response to lipopolysaccharide & 141 &  14 & 3.13 & 0.00 & 0.00 \\ 
  92 & GO:1903038 & negative regulation of leukocyte cell-ce... &  70 &  10 & 1.55 & 0.00 & 0.00 \\ 
  93 & GO:0002824 & positive regulation of adaptive immune r... &  56 &   9 & 1.24 & 0.00 & 0.00 \\ 
  94 & GO:0039529 & RIG-I signaling pathway &  12 &   5 & 0.27 & 0.00 & 0.00 \\ 
  95 & GO:0045824 & negative regulation of innate immune res... &  31 &   7 & 0.69 & 0.00 & 0.00 \\ 
  96 & GO:0050830 & defense response to Gram-positive bacter... &  45 &   8 & 1.00 & 0.00 & 0.00 \\ 
  97 & GO:0002819 & regulation of adaptive immune response &  93 &  11 & 2.06 & 0.00 & 0.00 \\ 
  98 & GO:0002253 & activation of immune response & 194 &  16 & 4.30 & 0.00 & 0.00 \\ 
  99 & GO:0002250 & adaptive immune response & 217 &  17 & 4.81 & 0.00 & 0.00 \\ 
  100 & GO:1900246 & positive regulation of RIG-I signaling p... &   7 &   4 & 0.16 & 0.00 & 0.00 \\ 
   \hline
\end{tabular}
\end{table}


  Cluster 1, MF, Mm.
  Overrepresented GO molecular function terms among mouse genes which have low abundance pre infection and 
  become more abundant in most day 3 and 5 samples from immunecompetent mice, but less in immune deficient Rag1-/- mice. 
  The upregulation trend is most clear in three first infection day 5 samples, which also form a separate
  sample cluster. On day 7, most of these genes are not differentially abundant, or below average, i.e., they
  seem to peak on day 5 p.i..\\
  % latex table generated in R 3.2.3 by xtable 1.8-2 package
% Fri Jun  3 15:12:43 2016
\begin{table}[ht]
\centering
\begin{tabular}{llrrrrr}
  \hline
GO.ID & Term & Annotated & Significant & Expected & p.value & adj.p \\ 
  \hline
GO:0016763 & transferase activity, transferring pento... &  38 &   7 & 0.77 & 0.00 & 0.00 \\ 
  GO:0003950 & NAD+ ADP-ribosyltransferase activity &  17 &   5 & 0.34 & 0.00 & 0.00 \\ 
  GO:0003725 & double-stranded RNA binding &  57 &   8 & 1.15 & 0.00 & 0.00 \\ 
  GO:0097367 & carbohydrate derivative binding & 1554 &  53 & 31.39 & 0.00 & 0.00 \\ 
  GO:0001730 & 2'-5'-oligoadenylate synthetase activity &   5 &   3 & 0.10 & 0.00 & 0.00 \\ 
  GO:0004298 & threonine-type endopeptidase activity &  18 &   4 & 0.36 & 0.00 & 0.00 \\ 
  GO:0070003 & threonine-type peptidase activity &  18 &   4 & 0.36 & 0.00 & 0.00 \\ 
  GO:0003692 & left-handed Z-DNA binding &   2 &   2 & 0.04 & 0.00 & 0.00 \\ 
  GO:0003726 & double-stranded RNA adenosine deaminase ... &   2 &   2 & 0.04 & 0.00 & 0.00 \\ 
  GO:0004833 & tryptophan 2,3-dioxygenase activity &   2 &   2 & 0.04 & 0.00 & 0.00 \\ 
  GO:0033754 & indoleamine 2,3-dioxygenase activity &   2 &   2 & 0.04 & 0.00 & 0.00 \\ 
  GO:0046980 & tapasin binding &   2 &   2 & 0.04 & 0.00 & 0.00 \\ 
  GO:0003823 & antigen binding &  58 &   6 & 1.17 & 0.00 & 0.01 \\ 
  GO:0023029 & MHC class Ib protein binding &   3 &   2 & 0.06 & 0.00 & 0.01 \\ 
  GO:0046978 & TAP1 binding &   3 &   2 & 0.06 & 0.00 & 0.01 \\ 
  GO:0046979 & TAP2 binding &   3 &   2 & 0.06 & 0.00 & 0.01 \\ 
  GO:0043168 & anion binding & 1880 &  56 & 37.97 & 0.00 & 0.01 \\ 
  GO:0003924 & GTPase activity & 170 &  10 & 3.43 & 0.00 & 0.01 \\ 
  GO:0030151 & molybdenum ion binding &   4 &   2 & 0.08 & 0.00 & 0.01 \\ 
  GO:0043167 & ion binding & 3732 &  96 & 75.38 & 0.00 & 0.01 \\ 
  GO:0023026 & MHC class II protein complex binding &  15 &   3 & 0.30 & 0.00 & 0.01 \\ 
  GO:0061133 & endopeptidase activator activity &   5 &   2 & 0.10 & 0.00 & 0.02 \\ 
  GO:0023023 & MHC protein complex binding &  17 &   3 & 0.34 & 0.00 & 0.02 \\ 
  GO:0005539 & glycosaminoglycan binding &  78 &   6 & 1.58 & 0.00 & 0.02 \\ 
  GO:0016701 & oxidoreductase activity, acting on singl... &  18 &   3 & 0.36 & 0.01 & 0.02 \\ 
  GO:0016702 & oxidoreductase activity, acting on singl... &  18 &   3 & 0.36 & 0.01 & 0.02 \\ 
  GO:0042287 & MHC protein binding &  18 &   3 & 0.36 & 0.01 & 0.02 \\ 
  GO:0016814 & hydrolase activity, acting on carbon-nit... &  19 &   3 & 0.38 & 0.01 & 0.02 \\ 
  GO:0017076 & purine nucleotide binding & 1372 &  41 & 27.71 & 0.01 & 0.02 \\ 
  GO:0032553 & ribonucleotide binding & 1375 &  41 & 27.77 & 0.01 & 0.02 \\ 
  GO:0035639 & purine ribonucleoside triphosphate bindi... & 1342 &  40 & 27.11 & 0.01 & 0.02 \\ 
  GO:0032550 & purine ribonucleoside binding & 1348 &  40 & 27.23 & 0.01 & 0.02 \\ 
  GO:0001883 & purine nucleoside binding & 1350 &  40 & 27.27 & 0.01 & 0.02 \\ 
  GO:0032549 & ribonucleoside binding & 1351 &  40 & 27.29 & 0.01 & 0.02 \\ 
  GO:0001882 & nucleoside binding & 1358 &  40 & 27.43 & 0.01 & 0.02 \\ 
  GO:0019239 & deaminase activity &  22 &   3 & 0.44 & 0.01 & 0.02 \\ 
  GO:0032555 & purine ribonucleotide binding & 1364 &  40 & 27.55 & 0.01 & 0.02 \\ 
   \hline
\end{tabular}
\end{table}

  
  Cluster 2, BP, Mm.
  Overrepresented GO biological process terms among mouse genes which are downregulated on day 7 p.i. and
  upregulated pre infection, i.e., these genes become more and more downregulated during infection.\\
  % latex table generated in R 3.2.3 by xtable 1.8-2 package
% Thu Apr 14 14:59:44 2016
\begin{table}[ht]
\centering
\begin{tabular}{llrrrrr}
  \hline
GO.ID & Term & Annotated & Significant & Expected & p.value & adj.p \\ 
  \hline
GO:0055114 & oxidation-reduction process & 650 &  28 & 9.76 & 0.00 & 0.00 \\ 
  GO:0044281 & small molecule metabolic process & 1261 &  42 & 18.94 & 0.00 & 0.00 \\ 
  GO:0006629 & lipid metabolic process & 796 &  29 & 11.95 & 0.00 & 0.00 \\ 
  GO:0044710 & single-organism metabolic process & 3313 &  76 & 49.76 & 0.00 & 0.00 \\ 
  GO:0006805 & xenobiotic metabolic process &  42 &   6 & 0.63 & 0.00 & 0.00 \\ 
  GO:0071466 & cellular response to xenobiotic stimulus &  46 &   6 & 0.69 & 0.00 & 0.00 \\ 
  GO:0009410 & response to xenobiotic stimulus &  51 &   6 & 0.77 & 0.00 & 0.00 \\ 
  GO:0006140 & regulation of nucleotide metabolic proce... & 103 &   8 & 1.55 & 0.00 & 0.00 \\ 
  GO:1901387 & positive regulation of voltage-gated cal... &   2 &   2 & 0.03 & 0.00 & 0.00 \\ 
  GO:0009117 & nucleotide metabolic process & 400 &  16 & 6.01 & 0.00 & 0.00 \\ 
  GO:0017144 & drug metabolic process &  23 &   4 & 0.35 & 0.00 & 0.00 \\ 
  GO:0019369 & arachidonic acid metabolic process &  24 &   4 & 0.36 & 0.00 & 0.00 \\ 
  GO:0006753 & nucleoside phosphate metabolic process & 408 &  16 & 6.13 & 0.00 & 0.00 \\ 
  GO:0008202 & steroid metabolic process & 169 &   9 & 2.54 & 0.00 & 0.01 \\ 
  GO:0055086 & nucleobase-containing small molecule met... & 443 &  16 & 6.65 & 0.00 & 0.01 \\ 
  GO:0015701 & bicarbonate transport &  14 &   3 & 0.21 & 0.00 & 0.01 \\ 
  GO:0042738 & exogenous drug catabolic process &  14 &   3 & 0.21 & 0.00 & 0.01 \\ 
  GO:0006631 & fatty acid metabolic process & 243 &  11 & 3.65 & 0.00 & 0.01 \\ 
  GO:0051289 & protein homotetramerization &  53 &   5 & 0.80 & 0.00 & 0.01 \\ 
  GO:0009992 & cellular water homeostasis &   4 &   2 & 0.06 & 0.00 & 0.01 \\ 
  GO:0032782 & bile acid secretion &   4 &   2 & 0.06 & 0.00 & 0.01 \\ 
  GO:0042737 & drug catabolic process &  15 &   3 & 0.23 & 0.00 & 0.01 \\ 
  GO:0006690 & icosanoid metabolic process &  55 &   5 & 0.83 & 0.00 & 0.01 \\ 
  GO:1901568 & fatty acid derivative metabolic process &  55 &   5 & 0.83 & 0.00 & 0.01 \\ 
  GO:0015850 & organic hydroxy compound transport & 116 &   7 & 1.74 & 0.00 & 0.01 \\ 
  GO:0043583 & ear development & 116 &   7 & 1.74 & 0.00 & 0.01 \\ 
  GO:0051262 & protein tetramerization &  87 &   6 & 1.31 & 0.00 & 0.01 \\ 
  GO:0055085 & transmembrane transport & 563 &  18 & 8.46 & 0.00 & 0.01 \\ 
  GO:0045989 & positive regulation of striated muscle c... &   5 &   2 & 0.08 & 0.00 & 0.01 \\ 
  GO:0042445 & hormone metabolic process &  89 &   6 & 1.34 & 0.00 & 0.01 \\ 
  GO:0051186 & cofactor metabolic process & 265 &  11 & 3.98 & 0.00 & 0.01 \\ 
  GO:0030104 & water homeostasis &  18 &   3 & 0.27 & 0.00 & 0.01 \\ 
  GO:0071333 & cellular response to glucose stimulus &  62 &   5 & 0.93 & 0.00 & 0.01 \\ 
  GO:0019637 & organophosphate metabolic process & 668 &  20 & 10.03 & 0.00 & 0.01 \\ 
  GO:0046903 & secretion & 625 &  19 & 9.39 & 0.00 & 0.01 \\ 
  GO:0071326 & cellular response to monosaccharide stim... &  64 &   5 & 0.96 & 0.00 & 0.01 \\ 
  GO:0071331 & cellular response to hexose stimulus &  64 &   5 & 0.96 & 0.00 & 0.01 \\ 
  GO:0019373 & epoxygenase P450 pathway &   6 &   2 & 0.09 & 0.00 & 0.01 \\ 
  GO:0019532 & oxalate transport &   6 &   2 & 0.09 & 0.00 & 0.01 \\ 
  GO:0048505 & regulation of timing of cell differentia... &   6 &   2 & 0.09 & 0.00 & 0.01 \\ 
  GO:0071322 & cellular response to carbohydrate stimul... &  68 &   5 & 1.02 & 0.00 & 0.01 \\ 
  GO:1900542 & regulation of purine nucleotide metaboli... &  98 &   6 & 1.47 & 0.00 & 0.01 \\ 
  GO:0050433 & regulation of catecholamine secretion &  21 &   3 & 0.32 & 0.00 & 0.01 \\ 
  GO:0015711 & organic anion transport & 207 &   9 & 3.11 & 0.00 & 0.01 \\ 
  GO:0016042 & lipid catabolic process & 172 &   8 & 2.58 & 0.00 & 0.01 \\ 
  GO:0003214 & cardiac left ventricle morphogenesis &   7 &   2 & 0.11 & 0.00 & 0.01 \\ 
  GO:0006833 & water transport &   7 &   2 & 0.11 & 0.00 & 0.01 \\ 
  GO:0032099 & negative regulation of appetite &   7 &   2 & 0.11 & 0.00 & 0.01 \\ 
  GO:1902358 & sulfate transmembrane transport &   7 &   2 & 0.11 & 0.00 & 0.01 \\ 
  GO:0001678 & cellular glucose homeostasis &  72 &   5 & 1.08 & 0.00 & 0.01 \\ 
  GO:0032787 & monocarboxylic acid metabolic process & 377 &  13 & 5.66 & 0.00 & 0.01 \\ 
  GO:0007200 & phospholipase C-activating G-protein cou... &  23 &   3 & 0.35 & 0.00 & 0.01 \\ 
  GO:0050432 & catecholamine secretion &  23 &   3 & 0.35 & 0.00 & 0.01 \\ 
  GO:0001676 & long-chain fatty acid metabolic process &  47 &   4 & 0.71 & 0.01 & 0.01 \\ 
  GO:0032105 & negative regulation of response to extra... &  24 &   3 & 0.36 & 0.01 & 0.01 \\ 
  GO:0032108 & negative regulation of response to nutri... &  24 &   3 & 0.36 & 0.01 & 0.01 \\ 
  GO:0055082 & cellular chemical homeostasis & 342 &  12 & 5.14 & 0.01 & 0.01 \\ 
  GO:0032612 & interleukin-1 production &  48 &   4 & 0.72 & 0.01 & 0.01 \\ 
  GO:0006820 & anion transport & 302 &  11 & 4.54 & 0.01 & 0.01 \\ 
  GO:0032096 & negative regulation of response to food &   8 &   2 & 0.12 & 0.01 & 0.01 \\ 
  GO:0033604 & negative regulation of catecholamine sec... &   8 &   2 & 0.12 & 0.01 & 0.01 \\ 
  GO:0040034 & regulation of development, heterochronic &   8 &   2 & 0.12 & 0.01 & 0.01 \\ 
  GO:0060081 & membrane hyperpolarization &   8 &   2 & 0.12 & 0.01 & 0.01 \\ 
  GO:0060979 & vasculogenesis involved in coronary vasc... &   8 &   2 & 0.12 & 0.01 & 0.01 \\ 
  GO:0006730 & one-carbon metabolic process &  25 &   3 & 0.38 & 0.01 & 0.01 \\ 
  GO:1901661 & quinone metabolic process &  25 &   3 & 0.38 & 0.01 & 0.01 \\ 
  GO:1901135 & carbohydrate derivative metabolic proces... & 677 &  19 & 10.17 & 0.01 & 0.01 \\ 
  GO:0034220 & ion transmembrane transport & 393 &  13 & 5.90 & 0.01 & 0.01 \\ 
  GO:0030004 & cellular monovalent inorganic cation hom... &  50 &   4 & 0.75 & 0.01 & 0.01 \\ 
  GO:0009187 & cyclic nucleotide metabolic process &  79 &   5 & 1.19 & 0.01 & 0.01 \\ 
  GO:0010817 & regulation of hormone levels & 267 &  10 & 4.01 & 0.01 & 0.01 \\ 
  GO:0008272 & sulfate transport &   9 &   2 & 0.14 & 0.01 & 0.01 \\ 
  GO:0032148 & activation of protein kinase B activity &   9 &   2 & 0.14 & 0.01 & 0.01 \\ 
  GO:0071875 & adrenergic receptor signaling pathway &   9 &   2 & 0.14 & 0.01 & 0.01 \\ 
  GO:0052652 & cyclic purine nucleotide metabolic proce... &  52 &   4 & 0.78 & 0.01 & 0.01 \\ 
  GO:0072521 & purine-containing compound metabolic pro... & 315 &  11 & 4.73 & 0.01 & 0.01 \\ 
  GO:0009190 & cyclic nucleotide biosynthetic process &  53 &   4 & 0.80 & 0.01 & 0.01 \\ 
  GO:0051937 & catecholamine transport &  28 &   3 & 0.42 & 0.01 & 0.01 \\ 
  GO:0019318 & hexose metabolic process & 154 &   7 & 2.31 & 0.01 & 0.01 \\ 
  GO:0045981 & positive regulation of nucleotide metabo... &  54 &   4 & 0.81 & 0.01 & 0.01 \\ 
  GO:1900544 & positive regulation of purine nucleotide... &  54 &   4 & 0.81 & 0.01 & 0.01 \\ 
  GO:0045980 & negative regulation of nucleotide metabo... &  29 &   3 & 0.44 & 0.01 & 0.01 \\ 
  GO:0002275 & myeloid cell activation involved in immu... &  55 &   4 & 0.83 & 0.01 & 0.01 \\ 
  GO:0032098 & regulation of appetite &  10 &   2 & 0.15 & 0.01 & 0.01 \\ 
  GO:0051923 & sulfation &  10 &   2 & 0.15 & 0.01 & 0.01 \\ 
  GO:0042493 & response to drug & 158 &   7 & 2.37 & 0.01 & 0.01 \\ 
   \hline
\end{tabular}
\end{table}


  Cluster 2, MF, Mm.
  Overrepresented GO molecular function terms among mouse genes which are downregulated on day 7 p.i. and
  upregulated pre infection, i.e., these genes become more and more downregulated during infection.\\
  % latex table generated in R 3.2.2 by xtable 1.8-2 package
% Fri Apr  8 11:16:05 2016
\begin{table}[ht]
\centering
\begin{tabular}{rllrrrrr}
  \hline
 & GO.ID & Term & Annotated & Significant & Expected & p.value & adj.p \\ 
  \hline
1 & GO:0016491 & oxidoreductase activity & 523 &  25 & 7.99 & 0.00 & 0.00 \\ 
  2 & GO:0016614 & oxidoreductase activity, acting on CH-OH... & 110 &  11 & 1.68 & 0.00 & 0.00 \\ 
  3 & GO:0016616 & oxidoreductase activity, acting on the C... &  92 &  10 & 1.41 & 0.00 & 0.00 \\ 
  4 & GO:0004090 & carbonyl reductase (NADPH) activity &   4 &   3 & 0.06 & 0.00 & 0.00 \\ 
  5 & GO:0004769 & steroid delta-isomerase activity &   2 &   2 & 0.03 & 0.00 & 0.00 \\ 
  6 & GO:0008390 & testosterone 16-alpha-hydroxylase activi... &   2 &   2 & 0.03 & 0.00 & 0.00 \\ 
  7 & GO:0033695 & oxidoreductase activity, acting on CH or... &   2 &   2 & 0.03 & 0.00 & 0.00 \\ 
  8 & GO:0034875 & caffeine oxidase activity &   2 &   2 & 0.03 & 0.00 & 0.00 \\ 
  9 & GO:0004089 & carbonate dehydratase activity &  10 &   3 & 0.15 & 0.00 & 0.00 \\ 
  10 & GO:0015106 & bicarbonate transmembrane transporter ac... &  11 &   3 & 0.17 & 0.00 & 0.01 \\ 
  11 & GO:0015168 & glycerol transmembrane transporter activ... &   3 &   2 & 0.05 & 0.00 & 0.01 \\ 
  12 & GO:0015254 & glycerol channel activity &   3 &   2 & 0.05 & 0.00 & 0.01 \\ 
  13 & GO:0008392 & arachidonic acid epoxygenase activity &  12 &   3 & 0.18 & 0.00 & 0.01 \\ 
  14 & GO:0008391 & arachidonic acid monooxygenase activity &  13 &   3 & 0.20 & 0.00 & 0.01 \\ 
  15 & GO:0016712 & oxidoreductase activity, acting on paire... &  14 &   3 & 0.21 & 0.00 & 0.01 \\ 
  16 & GO:0022838 & substrate-specific channel activity & 139 &   8 & 2.12 & 0.00 & 0.01 \\ 
  17 & GO:0015166 & polyol transmembrane transporter activit... &   4 &   2 & 0.06 & 0.00 & 0.01 \\ 
  18 & GO:0015250 & water channel activity &   4 &   2 & 0.06 & 0.00 & 0.01 \\ 
  19 & GO:0016229 & steroid dehydrogenase activity &  33 &   4 & 0.50 & 0.00 & 0.01 \\ 
  20 & GO:0052689 & carboxylic ester hydrolase activity & 111 &   7 & 1.70 & 0.00 & 0.01 \\ 
  21 & GO:0008395 & steroid hydroxylase activity &  16 &   3 & 0.24 & 0.00 & 0.01 \\ 
  22 & GO:0015301 & anion:anion antiporter activity &  16 &   3 & 0.24 & 0.00 & 0.01 \\ 
  23 & GO:0020037 & heme binding &  57 &   5 & 0.87 & 0.00 & 0.01 \\ 
  24 & GO:0016298 & lipase activity &  84 &   6 & 1.28 & 0.00 & 0.01 \\ 
  25 & GO:0015267 & channel activity & 146 &   8 & 2.23 & 0.00 & 0.01 \\ 
  26 & GO:0022803 & passive transmembrane transporter activi... & 146 &   8 & 2.23 & 0.00 & 0.01 \\ 
  27 & GO:0003824 & catalytic activity & 4065 &  81 & 62.10 & 0.00 & 0.01 \\ 
  28 & GO:0022857 & transmembrane transporter activity & 468 &  16 & 7.15 & 0.00 & 0.01 \\ 
  29 & GO:0003854 & 3-beta-hydroxy-delta5-steroid dehydrogen... &   5 &   2 & 0.08 & 0.00 & 0.01 \\ 
  30 & GO:0005372 & water transmembrane transporter activity &   5 &   2 & 0.08 & 0.00 & 0.01 \\ 
  31 & GO:0019825 & oxygen binding &  18 &   3 & 0.27 & 0.00 & 0.01 \\ 
  32 & GO:0008237 & metallopeptidase activity &  90 &   6 & 1.37 & 0.00 & 0.01 \\ 
  33 & GO:0050662 & coenzyme binding & 157 &   8 & 2.40 & 0.00 & 0.01 \\ 
  34 & GO:0046906 & tetrapyrrole binding &  64 &   5 & 0.98 & 0.00 & 0.01 \\ 
  35 & GO:0022892 & substrate-specific transporter activity & 531 &  17 & 8.11 & 0.00 & 0.01 \\ 
  36 & GO:0019531 & oxalate transmembrane transporter activi... &   6 &   2 & 0.09 & 0.00 & 0.01 \\ 
  37 & GO:0072341 & modified amino acid binding &  41 &   4 & 0.63 & 0.00 & 0.01 \\ 
  38 & GO:0010296 & prenylcysteine methylesterase activity &  21 &   3 & 0.32 & 0.00 & 0.01 \\ 
  39 & GO:0018731 & 1-oxa-2-oxocycloheptane lactonase activi... &  21 &   3 & 0.32 & 0.00 & 0.01 \\ 
  40 & GO:0018732 & sulfolactone hydrolase activity &  21 &   3 & 0.32 & 0.00 & 0.01 \\ 
  41 & GO:0018734 & butyrolactone hydrolase activity &  21 &   3 & 0.32 & 0.00 & 0.01 \\ 
  42 & GO:0034892 & endosulfan lactone lactonase activity &  21 &   3 & 0.32 & 0.00 & 0.01 \\ 
  43 & GO:0035460 & L-ascorbate 6-phosphate lactonase activi... &  21 &   3 & 0.32 & 0.00 & 0.01 \\ 
  44 & GO:0043905 & Ser-tRNA(Thr) hydrolase activity &  21 &   3 & 0.32 & 0.00 & 0.01 \\ 
  45 & GO:0043906 & Ala-tRNA(Pro) hydrolase activity &  21 &   3 & 0.32 & 0.00 & 0.01 \\ 
  46 & GO:0043907 & Cys-tRNA(Pro) hydrolase activity &  21 &   3 & 0.32 & 0.00 & 0.01 \\ 
  47 & GO:0043908 & Ser(Gly)-tRNA(Ala) hydrolase activity &  21 &   3 & 0.32 & 0.00 & 0.01 \\ 
  48 & GO:0047376 & all-trans-retinyl-palmitate hydrolase, a... &  21 &   3 & 0.32 & 0.00 & 0.01 \\ 
  49 & GO:0050253 & retinyl-palmitate esterase activity &  21 &   3 & 0.32 & 0.00 & 0.01 \\ 
  50 & GO:0052767 & mannosyl-oligosaccharide 1,6-alpha-manno... &  21 &   3 & 0.32 & 0.00 & 0.01 \\ 
  51 & GO:0052768 & mannosyl-oligosaccharide 1,3-alpha-manno... &  21 &   3 & 0.32 & 0.00 & 0.01 \\ 
  52 & GO:0080030 & methyl indole-3-acetate esterase activit... &  21 &   3 & 0.32 & 0.00 & 0.01 \\ 
  53 & GO:0080031 & methyl salicylate esterase activity &  21 &   3 & 0.32 & 0.00 & 0.01 \\ 
  54 & GO:0080032 & methyl jasmonate esterase activity &  21 &   3 & 0.32 & 0.00 & 0.01 \\ 
  55 & GO:0018733 & 3,4-dihydrocoumarin hydrolase activity &  22 &   3 & 0.34 & 0.00 & 0.01 \\ 
  56 & GO:0051723 & protein methylesterase activity &  22 &   3 & 0.34 & 0.00 & 0.01 \\ 
  57 & GO:0008271 & secondary active sulfate transmembrane t... &   7 &   2 & 0.11 & 0.00 & 0.01 \\ 
  58 & GO:0015665 & alcohol transmembrane transporter activi... &   7 &   2 & 0.11 & 0.00 & 0.01 \\ 
  59 & GO:0016725 & oxidoreductase activity, acting on CH or... &   7 &   2 & 0.11 & 0.00 & 0.01 \\ 
  60 & GO:0048037 & cofactor binding & 208 &   9 & 3.18 & 0.00 & 0.01 \\ 
  61 & GO:0004222 & metalloendopeptidase activity &  46 &   4 & 0.70 & 0.01 & 0.01 \\ 
  62 & GO:0022891 & substrate-specific transmembrane transpo... & 422 &  14 & 6.45 & 0.01 & 0.01 \\ 
  63 & GO:0005215 & transporter activity & 612 &  18 & 9.35 & 0.01 & 0.01 \\ 
  64 & GO:0004181 & metallocarboxypeptidase activity &   8 &   2 & 0.12 & 0.01 & 0.01 \\ 
  65 & GO:0015116 & sulfate transmembrane transporter activi... &   8 &   2 & 0.12 & 0.01 & 0.01 \\ 
  66 & GO:0016857 & racemase and epimerase activity, acting ... &   8 &   2 & 0.12 & 0.01 & 0.01 \\ 
  67 & GO:0070700 & BMP receptor binding &   8 &   2 & 0.12 & 0.01 & 0.01 \\ 
  68 & GO:0016829 & lyase activity & 144 &   7 & 2.20 & 0.01 & 0.01 \\ 
  69 & GO:0016788 & hydrolase activity, acting on ester bond... & 572 &  17 & 8.74 & 0.01 & 0.01 \\ 
  70 & GO:0008083 & growth factor activity &  52 &   4 & 0.79 & 0.01 & 0.01 \\ 
  71 & GO:0033764 & steroid dehydrogenase activity, acting o... &  29 &   3 & 0.44 & 0.01 & 0.01 \\ 
  72 & GO:0016208 & AMP binding &  10 &   2 & 0.15 & 0.01 & 0.01 \\ 
  73 & GO:0044548 & S100 protein binding &  10 &   2 & 0.15 & 0.01 & 0.01 \\ 
  74 & GO:0016836 & hydro-lyase activity &  55 &   4 & 0.84 & 0.01 & 0.01 \\ 
   \hline
\end{tabular}
\end{table}

 
  Cluster 3, BP, Mm.
  Overrepresented GO biological process terms among mouse genes which are upregulated pre infection and
  downregulated on day 7 p.i., except in one NMRI second infection sample, possibly indicating no or
  poor infection in this sample. As in cluster 2, these mRNAs are most highly abundant before infection and
  become less abundant over the course of infection.\\
  % latex table generated in R 3.2.3 by xtable 1.8-2 package
% Thu Apr 14 15:01:22 2016
\begin{table}[ht]
\centering
\begin{tabular}{llrrrrr}
  \hline
GO.ID & Term & Annotated & Significant & Expected & p.value & adj.p \\ 
  \hline
GO:0006805 & xenobiotic metabolic process &  42 &  18 & 1.51 & 0.00 & 0.00 \\ 
  GO:0006082 & organic acid metabolic process & 661 &  68 & 23.81 & 0.00 & 0.00 \\ 
  GO:0044281 & small molecule metabolic process & 1261 & 101 & 45.43 & 0.00 & 0.00 \\ 
  GO:0032787 & monocarboxylic acid metabolic process & 377 &  49 & 13.58 & 0.00 & 0.00 \\ 
  GO:0071466 & cellular response to xenobiotic stimulus &  46 &  18 & 1.66 & 0.00 & 0.00 \\ 
  GO:0009813 & flavonoid biosynthetic process &  13 &  11 & 0.47 & 0.00 & 0.00 \\ 
  GO:0052696 & flavonoid glucuronidation &  13 &  11 & 0.47 & 0.00 & 0.00 \\ 
  GO:0043436 & oxoacid metabolic process & 652 &  66 & 23.49 & 0.00 & 0.00 \\ 
  GO:0019752 & carboxylic acid metabolic process & 612 &  63 & 22.05 & 0.00 & 0.00 \\ 
  GO:0009812 & flavonoid metabolic process &  14 &  11 & 0.50 & 0.00 & 0.00 \\ 
  GO:0052695 & cellular glucuronidation &  14 &  11 & 0.50 & 0.00 & 0.00 \\ 
  GO:0009410 & response to xenobiotic stimulus &  51 &  18 & 1.84 & 0.00 & 0.00 \\ 
  GO:0006063 & uronic acid metabolic process &  15 &  11 & 0.54 & 0.00 & 0.00 \\ 
  GO:0019585 & glucuronate metabolic process &  15 &  11 & 0.54 & 0.00 & 0.00 \\ 
  GO:0052697 & xenobiotic glucuronidation &   8 &   8 & 0.29 & 0.00 & 0.00 \\ 
  GO:0006629 & lipid metabolic process & 796 &  66 & 28.68 & 0.00 & 0.00 \\ 
  GO:0009719 & response to endogenous stimulus & 729 &  60 & 26.26 & 0.00 & 0.00 \\ 
  GO:0071495 & cellular response to endogenous stimulus & 561 &  50 & 20.21 & 0.00 & 0.00 \\ 
  GO:0009725 & response to hormone & 385 &  37 & 13.87 & 0.00 & 0.00 \\ 
  GO:0006631 & fatty acid metabolic process & 243 &  28 & 8.75 & 0.00 & 0.00 \\ 
  GO:0044255 & cellular lipid metabolic process & 604 &  49 & 21.76 & 0.00 & 0.00 \\ 
  GO:0055114 & oxidation-reduction process & 650 &  50 & 23.42 & 0.00 & 0.00 \\ 
  GO:0042445 & hormone metabolic process &  89 &  15 & 3.21 & 0.00 & 0.00 \\ 
  GO:0032870 & cellular response to hormone stimulus & 278 &  28 & 10.02 & 0.00 & 0.00 \\ 
  GO:0044710 & single-organism metabolic process & 3313 & 163 & 119.36 & 0.00 & 0.00 \\ 
  GO:0034754 & cellular hormone metabolic process &  43 &  10 & 1.55 & 0.00 & 0.00 \\ 
  GO:0005996 & monosaccharide metabolic process & 177 &  20 & 6.38 & 0.00 & 0.00 \\ 
  GO:0042178 & xenobiotic catabolic process &   9 &   5 & 0.32 & 0.00 & 0.00 \\ 
  GO:0042537 & benzene-containing compound metabolic pr... &  15 &   6 & 0.54 & 0.00 & 0.00 \\ 
  GO:0006790 & sulfur compound metabolic process & 200 &  21 & 7.21 & 0.00 & 0.00 \\ 
  GO:0019395 & fatty acid oxidation &  66 &  11 & 2.38 & 0.00 & 0.00 \\ 
  GO:0034440 & lipid oxidation &  67 &  11 & 2.41 & 0.00 & 0.00 \\ 
  GO:0044699 & single-organism process & 7965 & 325 & 286.96 & 0.00 & 0.00 \\ 
  GO:0042221 & response to chemical & 1868 &  99 & 67.30 & 0.00 & 0.00 \\ 
  GO:1901615 & organic hydroxy compound metabolic proce... & 299 &  26 & 10.77 & 0.00 & 0.00 \\ 
  GO:0010817 & regulation of hormone levels & 267 &  24 & 9.62 & 0.00 & 0.00 \\ 
  GO:0016042 & lipid catabolic process & 172 &  18 & 6.20 & 0.00 & 0.00 \\ 
  GO:0006811 & ion transport & 785 &  50 & 28.28 & 0.00 & 0.00 \\ 
  GO:0022600 & digestive system process &  51 &   9 & 1.84 & 0.00 & 0.00 \\ 
  GO:0044242 & cellular lipid catabolic process & 117 &  14 & 4.22 & 0.00 & 0.00 \\ 
  GO:0042738 & exogenous drug catabolic process &  14 &   5 & 0.50 & 0.00 & 0.00 \\ 
  GO:0051386 & regulation of neurotrophin TRK receptor ... &   8 &   4 & 0.29 & 0.00 & 0.00 \\ 
  GO:0044344 & cellular response to fibroblast growth f... &  42 &   8 & 1.51 & 0.00 & 0.00 \\ 
  GO:0008202 & steroid metabolic process & 169 &  17 & 6.09 & 0.00 & 0.00 \\ 
  GO:0017144 & drug metabolic process &  23 &   6 & 0.83 & 0.00 & 0.00 \\ 
  GO:0042737 & drug catabolic process &  15 &   5 & 0.54 & 0.00 & 0.00 \\ 
  GO:0005975 & carbohydrate metabolic process & 481 &  34 & 17.33 & 0.00 & 0.00 \\ 
  GO:0072329 & monocarboxylic acid catabolic process &  68 &  10 & 2.45 & 0.00 & 0.00 \\ 
  GO:0071774 & response to fibroblast growth factor &  44 &   8 & 1.59 & 0.00 & 0.00 \\ 
  GO:0007586 & digestion &  56 &   9 & 2.02 & 0.00 & 0.00 \\ 
  GO:0072001 & renal system development & 141 &  15 & 5.08 & 0.00 & 0.00 \\ 
  GO:0044539 & long-chain fatty acid import &   4 &   3 & 0.14 & 0.00 & 0.00 \\ 
  GO:0051387 & negative regulation of neurotrophin TRK ... &   4 &   3 & 0.14 & 0.00 & 0.00 \\ 
  GO:0042493 & response to drug & 158 &  16 & 5.69 & 0.00 & 0.00 \\ 
  GO:0006672 & ceramide metabolic process &  58 &   9 & 2.09 & 0.00 & 0.00 \\ 
  GO:0008543 & fibroblast growth factor receptor signal... &  35 &   7 & 1.26 & 0.00 & 0.00 \\ 
  GO:0044723 & single-organism carbohydrate metabolic p... & 435 &  31 & 15.67 & 0.00 & 0.00 \\ 
  GO:0007267 & cell-cell signaling & 456 &  32 & 16.43 & 0.00 & 0.00 \\ 
  GO:0070887 & cellular response to chemical stimulus & 1330 &  72 & 47.92 & 0.00 & 0.00 \\ 
  GO:0001822 & kidney development & 131 &  14 & 4.72 & 0.00 & 0.00 \\ 
  GO:0006040 & amino sugar metabolic process &  26 &   6 & 0.94 & 0.00 & 0.00 \\ 
  GO:0034762 & regulation of transmembrane transport & 180 &  17 & 6.48 & 0.00 & 0.00 \\ 
  GO:0042572 & retinol metabolic process &  10 &   4 & 0.36 & 0.00 & 0.00 \\ 
  GO:0043269 & regulation of ion transport & 306 &  24 & 11.02 & 0.00 & 0.00 \\ 
  GO:0009062 & fatty acid catabolic process &  61 &   9 & 2.20 & 0.00 & 0.00 \\ 
  GO:0044282 & small molecule catabolic process & 183 &  17 & 6.59 & 0.00 & 0.00 \\ 
  GO:0001655 & urogenital system development & 168 &  16 & 6.05 & 0.00 & 0.00 \\ 
  GO:0022603 & regulation of anatomical structure morph... & 572 &  37 & 20.61 & 0.00 & 0.00 \\ 
  GO:0044763 & single-organism cellular process & 7155 & 291 & 257.78 & 0.00 & 0.00 \\ 
  GO:0006637 & acyl-CoA metabolic process &  52 &   8 & 1.87 & 0.00 & 0.00 \\ 
  GO:0035383 & thioester metabolic process &  52 &   8 & 1.87 & 0.00 & 0.00 \\ 
  GO:0006732 & coenzyme metabolic process & 227 &  19 & 8.18 & 0.00 & 0.00 \\ 
  GO:0016054 & organic acid catabolic process & 126 &  13 & 4.54 & 0.00 & 0.00 \\ 
  GO:0046395 & carboxylic acid catabolic process & 126 &  13 & 4.54 & 0.00 & 0.00 \\ 
  GO:0015698 & inorganic anion transport &  67 &   9 & 2.41 & 0.00 & 0.00 \\ 
  GO:0007169 & transmembrane receptor protein tyrosine ... & 324 &  24 & 11.67 & 0.00 & 0.00 \\ 
  GO:0009308 & amine metabolic process &  82 &  10 & 2.95 & 0.00 & 0.00 \\ 
  GO:0048011 & neurotrophin TRK receptor signaling path... &  22 &   5 & 0.79 & 0.00 & 0.00 \\ 
  GO:0030258 & lipid modification & 134 &  13 & 4.83 & 0.00 & 0.00 \\ 
  GO:0034765 & regulation of ion transmembrane transpor... & 168 &  15 & 6.05 & 0.00 & 0.00 \\ 
  GO:0055085 & transmembrane transport & 563 &  35 & 20.28 & 0.00 & 0.00 \\ 
  GO:0006576 & cellular biogenic amine metabolic proces... &  59 &   8 & 2.13 & 0.00 & 0.00 \\ 
  GO:0032941 & secretion by tissue &  46 &   7 & 1.66 & 0.00 & 0.00 \\ 
  GO:0043537 & negative regulation of blood vessel endo... &  14 &   4 & 0.50 & 0.00 & 0.00 \\ 
  GO:0060602 & branch elongation of an epithelium &  14 &   4 & 0.50 & 0.00 & 0.00 \\ 
  GO:0018879 & biphenyl metabolic process &   2 &   2 & 0.07 & 0.00 & 0.00 \\ 
  GO:0046168 & glycerol-3-phosphate catabolic process &   2 &   2 & 0.07 & 0.00 & 0.00 \\ 
  GO:0061031 & endodermal digestive tract morphogenesis &   2 &   2 & 0.07 & 0.00 & 0.00 \\ 
  GO:0097089 & methyl-branched fatty acid metabolic pro... &   2 &   2 & 0.07 & 0.00 & 0.00 \\ 
  GO:0097623 & potassium ion export across plasma membr... &   2 &   2 & 0.07 & 0.00 & 0.00 \\ 
  GO:0001676 & long-chain fatty acid metabolic process &  47 &   7 & 1.69 & 0.00 & 0.00 \\ 
  GO:0007268 & synaptic transmission & 245 &  19 & 8.83 & 0.00 & 0.00 \\ 
  GO:0071310 & cellular response to organic substance & 1039 &  56 & 37.43 & 0.00 & 0.00 \\ 
  GO:0019755 & one-carbon compound transport &   7 &   3 & 0.25 & 0.00 & 0.00 \\ 
  GO:0060736 & prostate gland growth &   7 &   3 & 0.25 & 0.00 & 0.00 \\ 
  GO:0090331 & negative regulation of platelet aggregat... &   7 &   3 & 0.25 & 0.00 & 0.00 \\ 
  GO:0008610 & lipid biosynthetic process & 383 &  26 & 13.80 & 0.00 & 0.00 \\ 
  GO:0009755 & hormone-mediated signaling pathway &  61 &   8 & 2.20 & 0.00 & 0.00 \\ 
  GO:0007167 & enzyme linked receptor protein signaling... & 487 &  31 & 17.55 & 0.00 & 0.00 \\ 
  GO:0016266 & O-glycan processing &  15 &   4 & 0.54 & 0.00 & 0.00 \\ 
   \hline
\end{tabular}
\end{table}

  
  Cluster 3, MF, Mm.
  Overrepresented GO molecular function terms among mouse genes which are upregulated pre infection and
  downregulated on day 7 p.i., except in one NMRI second infection sample, possibly indicating no or
  poor infection in this sample. As in cluster 2, these mRNAs are most highly abundant before infection and
  become less abundant over the course of infection.\\
  % latex table generated in R 3.2.3 by xtable 1.8-2 package
% Thu Mar 24 05:57:57 2016
\begin{table}[ht]
\centering
\begin{tabular}{rllrrrrr}
  \hline
 & GO.ID & Term & Annotated & Significant & Expected & p.value & adj.p \\ 
  \hline
1 & GO:0016491 & oxidoreductase activity & 481 &  24 & 6.91 & 0.00 & 0.00 \\ 
  2 & GO:0048037 & cofactor binding & 200 &  14 & 2.87 & 0.00 & 0.00 \\ 
  3 & GO:0050662 & coenzyme binding & 149 &  11 & 2.14 & 0.00 & 0.00 \\ 
  4 & GO:0008392 & arachidonic acid epoxygenase activity &  13 &   4 & 0.19 & 0.00 & 0.00 \\ 
  5 & GO:0008391 & arachidonic acid monooxygenase activity &  14 &   4 & 0.20 & 0.00 & 0.00 \\ 
  6 & GO:0003824 & catalytic activity & 3807 &  79 & 54.70 & 0.00 & 0.00 \\ 
  7 & GO:0016712 & oxidoreductase activity, acting on paire... &  15 &   4 & 0.22 & 0.00 & 0.00 \\ 
  8 & GO:0008237 & metallopeptidase activity &  93 &   8 & 1.34 & 0.00 & 0.00 \\ 
  9 & GO:0004222 & metalloendopeptidase activity &  50 &   6 & 0.72 & 0.00 & 0.00 \\ 
  10 & GO:0008395 & steroid hydroxylase activity &  17 &   4 & 0.24 & 0.00 & 0.00 \\ 
  11 & GO:0019825 & oxygen binding &  18 &   4 & 0.26 & 0.00 & 0.00 \\ 
  12 & GO:0008430 & selenium binding &   9 &   3 & 0.13 & 0.00 & 0.00 \\ 
  13 & GO:0016863 & intramolecular oxidoreductase activity, ... &  10 &   3 & 0.14 & 0.00 & 0.00 \\ 
  14 & GO:0051287 & NAD binding &  44 &   5 & 0.63 & 0.00 & 0.00 \\ 
  15 & GO:0015106 & bicarbonate transmembrane transporter ac... &  11 &   3 & 0.16 & 0.00 & 0.00 \\ 
  16 & GO:0004062 & aryl sulfotransferase activity &   3 &   2 & 0.04 & 0.00 & 0.00 \\ 
  17 & GO:0042392 & sphingosine-1-phosphate phosphatase acti... &   3 &   2 & 0.04 & 0.00 & 0.00 \\ 
  18 & GO:0016614 & oxidoreductase activity, acting on CH-OH... & 103 &   7 & 1.48 & 0.00 & 0.00 \\ 
  19 & GO:0020037 & heme binding &  54 &   5 & 0.78 & 0.00 & 0.01 \\ 
  20 & GO:0016782 & transferase activity, transferring sulfu... &  32 &   4 & 0.46 & 0.00 & 0.01 \\ 
  21 & GO:0004165 & dodecenoyl-CoA delta-isomerase activity &   4 &   2 & 0.06 & 0.00 & 0.01 \\ 
  22 & GO:0050694 & galactose 3-O-sulfotransferase activity &  16 &   3 & 0.23 & 0.00 & 0.01 \\ 
  23 & GO:0016616 & oxidoreductase activity, acting on the C... &  86 &   6 & 1.24 & 0.00 & 0.01 \\ 
  24 & GO:0015301 & anion:anion antiporter activity &  17 &   3 & 0.24 & 0.00 & 0.01 \\ 
  25 & GO:0046906 & tetrapyrrole binding &  61 &   5 & 0.88 & 0.00 & 0.01 \\ 
  26 & GO:1901681 & sulfur compound binding & 129 &   7 & 1.85 & 0.00 & 0.01 \\ 
  27 & GO:0015245 & fatty acid transporter activity &   6 &   2 & 0.09 & 0.00 & 0.01 \\ 
  28 & GO:0019531 & oxalate transmembrane transporter activi... &   6 &   2 & 0.09 & 0.00 & 0.01 \\ 
  29 & GO:0008146 & sulfotransferase activity &  21 &   3 & 0.30 & 0.00 & 0.01 \\ 
  30 & GO:0008233 & peptidase activity & 344 &  12 & 4.94 & 0.00 & 0.01 \\ 
  31 & GO:0008271 & secondary active sulfate transmembrane t... &   7 &   2 & 0.10 & 0.00 & 0.01 \\ 
  32 & GO:0042605 & peptide antigen binding &  24 &   3 & 0.34 & 0.00 & 0.01 \\ 
  33 & GO:0004467 & long-chain fatty acid-CoA ligase activit... &   8 &   2 & 0.11 & 0.01 & 0.01 \\ 
  34 & GO:0015116 & sulfate transmembrane transporter activi... &   8 &   2 & 0.11 & 0.01 & 0.01 \\ 
  35 & GO:0070330 & aromatase activity &   8 &   2 & 0.11 & 0.01 & 0.01 \\ 
  36 & GO:0070700 & BMP receptor binding &   8 &   2 & 0.11 & 0.01 & 0.01 \\ 
  37 & GO:0009055 & electron carrier activity &  50 &   4 & 0.72 & 0.01 & 0.01 \\ 
  38 & GO:0005179 & hormone activity &  26 &   3 & 0.37 & 0.01 & 0.01 \\ 
  39 & GO:0016877 & ligase activity, forming carbon-sulfur b... &  26 &   3 & 0.37 & 0.01 & 0.01 \\ 
  40 & GO:0004497 & monooxygenase activity &  52 &   4 & 0.75 & 0.01 & 0.01 \\ 
  41 & GO:0004175 & endopeptidase activity & 235 &   9 & 3.38 & 0.01 & 0.01 \\ 
  42 & GO:0004774 & succinate-CoA ligase activity &   9 &   2 & 0.13 & 0.01 & 0.01 \\ 
  43 & GO:0004872 & receptor activity & 417 &  13 & 5.99 & 0.01 & 0.01 \\ 
  44 & GO:0031406 & carboxylic acid binding & 119 &   6 & 1.71 & 0.01 & 0.01 \\ 
  45 & GO:0043177 & organic acid binding & 119 &   6 & 1.71 & 0.01 & 0.01 \\ 
  46 & GO:0015103 & inorganic anion transmembrane transporte... &  54 &   4 & 0.78 & 0.01 & 0.01 \\ 
  47 & GO:0019199 & transmembrane receptor protein kinase ac... &  54 &   4 & 0.78 & 0.01 & 0.01 \\ 
  48 & GO:0042974 & retinoic acid receptor binding &  29 &   3 & 0.42 & 0.01 & 0.01 \\ 
  49 & GO:0046872 & metal ion binding & 2232 &  45 & 32.07 & 0.01 & 0.01 \\ 
  50 & GO:0050660 & flavin adenine dinucleotide binding &  56 &   4 & 0.80 & 0.01 & 0.01 \\ 
  51 & GO:0033612 & receptor serine/threonine kinase binding &  10 &   2 & 0.14 & 0.01 & 0.01 \\ 
  52 & GO:0070696 & transmembrane receptor protein serine/th... &  10 &   2 & 0.14 & 0.01 & 0.01 \\ 
  53 & GO:0005102 & receptor binding & 838 &  21 & 12.04 & 0.01 & 0.01 \\ 
  54 & GO:0038023 & signaling receptor activity & 334 &  11 & 4.80 & 0.01 & 0.01 \\ 
  55 & GO:0005254 & chloride channel activity &  30 &   3 & 0.43 & 0.01 & 0.01 \\ 
  56 & GO:0051536 & iron-sulfur cluster binding &  57 &   4 & 0.82 & 0.01 & 0.01 \\ 
  57 & GO:0051540 & metal cluster binding &  57 &   4 & 0.82 & 0.01 & 0.01 \\ 
  58 & GO:0070011 & peptidase activity, acting on L-amino ac... & 337 &  11 & 4.84 & 0.01 & 0.01 \\ 
   \hline
\end{tabular}
\end{table}

  
  Cluster 4, BP, Mm.
  Overrepresented GO biological process terms among mouse genes which are highly abundant on day 7 p.i.. 
  There is overall no clear difference between pre infection and days 3 or 5 among these genesi. A slight
  tendency to higher abundance in second infection samples can be seen among these genes.\\
  % latex table generated in R 3.2.3 by xtable 1.8-2 package
% Fri Jun  3 15:15:21 2016
\begin{table}[ht]
\centering
\begin{tabular}{llrrrrr}
  \hline
GO.ID & Term & Annotated & Significant & Expected & p.value & adj.p \\ 
  \hline
GO:0002376 & immune system process & 1406 & 126 & 41.64 & 0.00 & 0.00 \\ 
  GO:0006955 & immune response & 713 &  85 & 21.12 & 0.00 & 0.00 \\ 
  GO:0050896 & response to stimulus & 4441 & 229 & 131.52 & 0.00 & 0.00 \\ 
  GO:0006954 & inflammatory response & 327 &  53 & 9.68 & 0.00 & 0.00 \\ 
  GO:0044700 & single organism signaling & 2975 & 176 & 88.11 & 0.00 & 0.00 \\ 
  GO:0006952 & defense response & 755 &  80 & 22.36 & 0.00 & 0.00 \\ 
  GO:0023052 & signaling & 2981 & 176 & 88.28 & 0.00 & 0.00 \\ 
  GO:0007165 & signal transduction & 2773 & 167 & 82.12 & 0.00 & 0.00 \\ 
  GO:0001775 & cell activation & 589 &  69 & 17.44 & 0.00 & 0.00 \\ 
  GO:0007154 & cell communication & 3137 & 178 & 92.90 & 0.00 & 0.00 \\ 
  GO:0022610 & biological adhesion & 800 &  79 & 23.69 & 0.00 & 0.00 \\ 
  GO:0002684 & positive regulation of immune system pro... & 494 &  61 & 14.63 & 0.00 & 0.00 \\ 
  GO:0007155 & cell adhesion & 791 &  78 & 23.43 & 0.00 & 0.00 \\ 
  GO:0002682 & regulation of immune system process & 774 &  76 & 22.92 & 0.00 & 0.00 \\ 
  GO:0001816 & cytokine production & 397 &  53 & 11.76 & 0.00 & 0.00 \\ 
  GO:0065007 & biological regulation & 6468 & 274 & 191.55 & 0.00 & 0.00 \\ 
  GO:0048518 & positive regulation of biological proces... & 3404 & 182 & 100.81 & 0.00 & 0.00 \\ 
  GO:0045321 & leukocyte activation & 515 &  60 & 15.25 & 0.00 & 0.00 \\ 
  GO:0040011 & locomotion & 903 &  79 & 26.74 & 0.00 & 0.00 \\ 
  GO:0098602 & single organism cell adhesion & 516 &  58 & 15.28 & 0.00 & 0.00 \\ 
  GO:0051239 & regulation of multicellular organismal p... & 1538 & 108 & 45.55 & 0.00 & 0.00 \\ 
  GO:0007159 & leukocyte cell-cell adhesion & 332 &  46 & 9.83 & 0.00 & 0.00 \\ 
  GO:0016337 & single organismal cell-cell adhesion & 474 &  55 & 14.04 & 0.00 & 0.00 \\ 
  GO:0051716 & cellular response to stimulus & 3741 & 189 & 110.79 & 0.00 & 0.00 \\ 
  GO:0098609 & cell-cell adhesion & 507 &  56 & 15.01 & 0.00 & 0.00 \\ 
  GO:0044707 & single-multicellular organism process & 3409 & 176 & 100.96 & 0.00 & 0.00 \\ 
  GO:0050789 & regulation of biological process & 6161 & 259 & 182.46 & 0.00 & 0.00 \\ 
  GO:0048870 & cell motility & 784 &  70 & 23.22 & 0.00 & 0.00 \\ 
  GO:0051674 & localization of cell & 784 &  70 & 23.22 & 0.00 & 0.00 \\ 
  GO:0032501 & multicellular organismal process & 3487 & 177 & 103.27 & 0.00 & 0.00 \\ 
  GO:0050794 & regulation of cellular process & 5853 & 249 & 173.34 & 0.00 & 0.00 \\ 
  GO:0016477 & cell migration & 740 &  67 & 21.92 & 0.00 & 0.00 \\ 
  GO:0001817 & regulation of cytokine production & 344 &  44 & 10.19 & 0.00 & 0.00 \\ 
  GO:0034109 & homotypic cell-cell adhesion & 361 &  45 & 10.69 & 0.00 & 0.00 \\ 
  GO:0048583 & regulation of response to stimulus & 2109 & 126 & 62.46 & 0.00 & 0.00 \\ 
  GO:0048584 & positive regulation of response to stimu... & 1130 &  85 & 33.47 & 0.00 & 0.00 \\ 
  GO:0050865 & regulation of cell activation & 319 &  42 & 9.45 & 0.00 & 0.00 \\ 
  GO:0048522 & positive regulation of cellular process & 3048 & 160 & 90.27 & 0.00 & 0.00 \\ 
  GO:0046649 & lymphocyte activation & 435 &  49 & 12.88 & 0.00 & 0.00 \\ 
  GO:0035556 & intracellular signal transduction & 1570 & 103 & 46.50 & 0.00 & 0.00 \\ 
  GO:0002694 & regulation of leukocyte activation & 301 &  40 & 8.91 & 0.00 & 0.00 \\ 
  GO:0070486 & leukocyte aggregation & 314 &  40 & 9.30 & 0.00 & 0.00 \\ 
  GO:1902531 & regulation of intracellular signal trans... & 991 &  76 & 29.35 & 0.00 & 0.00 \\ 
  GO:0050900 & leukocyte migration & 164 &  29 & 4.86 & 0.00 & 0.00 \\ 
  GO:0009605 & response to external stimulus & 1176 &  83 & 34.83 & 0.00 & 0.00 \\ 
  GO:0030155 & regulation of cell adhesion & 414 &  45 & 12.26 & 0.00 & 0.00 \\ 
  GO:0051249 & regulation of lymphocyte activation & 259 &  35 & 7.67 & 0.00 & 0.00 \\ 
  GO:0006928 & movement of cell or subcellular componen... & 992 &  74 & 29.38 & 0.00 & 0.00 \\ 
  GO:0001819 & positive regulation of cytokine producti... & 232 &  33 & 6.87 & 0.00 & 0.00 \\ 
  GO:0006950 & response to stress & 2191 & 123 & 64.89 & 0.00 & 0.00 \\ 
  GO:0060326 & cell chemotaxis & 145 &  26 & 4.29 & 0.00 & 0.00 \\ 
  GO:0050793 & regulation of developmental process & 1367 &  89 & 40.48 & 0.00 & 0.00 \\ 
  GO:0023051 & regulation of signaling & 1746 & 104 & 51.71 & 0.00 & 0.00 \\ 
  GO:0044763 & single-organism cellular process & 7155 & 274 & 211.90 & 0.00 & 0.00 \\ 
  GO:0006935 & chemotaxis & 270 &  34 & 8.00 & 0.00 & 0.00 \\ 
  GO:0008219 & cell death & 1372 &  88 & 40.63 & 0.00 & 0.00 \\ 
  GO:0042330 & taxis & 271 &  34 & 8.03 & 0.00 & 0.00 \\ 
  GO:0065008 & regulation of biological quality & 1978 & 112 & 58.58 & 0.00 & 0.00 \\ 
  GO:0016265 & death & 1376 &  88 & 40.75 & 0.00 & 0.00 \\ 
  GO:0070887 & cellular response to chemical stimulus & 1330 &  86 & 39.39 & 0.00 & 0.00 \\ 
  GO:0042110 & T cell activation & 305 &  36 & 9.03 & 0.00 & 0.00 \\ 
  GO:0070489 & T cell aggregation & 305 &  36 & 9.03 & 0.00 & 0.00 \\ 
  GO:0010646 & regulation of cell communication & 1781 & 104 & 52.74 & 0.00 & 0.00 \\ 
  GO:0071593 & lymphocyte aggregation & 307 &  36 & 9.09 & 0.00 & 0.00 \\ 
  GO:0009966 & regulation of signal transduction & 1557 &  95 & 46.11 & 0.00 & 0.00 \\ 
  GO:0051240 & positive regulation of multicellular org... & 890 &  66 & 26.36 & 0.00 & 0.00 \\ 
  GO:0048731 & system development & 2399 & 126 & 71.05 & 0.00 & 0.00 \\ 
  GO:0048856 & anatomical structure development & 2919 & 144 & 86.45 & 0.00 & 0.00 \\ 
  GO:0007166 & cell surface receptor signaling pathway & 1300 &  83 & 38.50 & 0.00 & 0.00 \\ 
  GO:0044699 & single-organism process & 7965 & 292 & 235.89 & 0.00 & 0.00 \\ 
  GO:0040017 & positive regulation of locomotion & 276 &  33 & 8.17 & 0.00 & 0.00 \\ 
  GO:0030154 & cell differentiation & 2151 & 116 & 63.70 & 0.00 & 0.00 \\ 
  GO:0002520 & immune system development & 623 &  52 & 18.45 & 0.00 & 0.00 \\ 
  GO:0002440 & production of molecular mediator of immu... & 111 &  21 & 3.29 & 0.00 & 0.00 \\ 
  GO:0048534 & hematopoietic or lymphoid organ developm... & 586 &  50 & 17.35 & 0.00 & 0.00 \\ 
  GO:0002521 & leukocyte differentiation & 347 &  37 & 10.28 & 0.00 & 0.00 \\ 
  GO:0050663 & cytokine secretion & 101 &  20 & 2.99 & 0.00 & 0.00 \\ 
  GO:0030595 & leukocyte chemotaxis & 103 &  20 & 3.05 & 0.00 & 0.00 \\ 
  GO:1902533 & positive regulation of intracellular sig... & 556 &  48 & 16.47 & 0.00 & 0.00 \\ 
  GO:0050867 & positive regulation of cell activation & 195 &  27 & 5.77 & 0.00 & 0.00 \\ 
  GO:0042221 & response to chemical & 1868 & 104 & 55.32 & 0.00 & 0.00 \\ 
  GO:0002250 & adaptive immune response & 226 &  29 & 6.69 & 0.00 & 0.00 \\ 
  GO:0097529 & myeloid leukocyte migration &  94 &  19 & 2.78 & 0.00 & 0.00 \\ 
  GO:0030335 & positive regulation of cell migration & 258 &  31 & 7.64 & 0.00 & 0.00 \\ 
  GO:0044767 & single-organism developmental process & 3254 & 153 & 96.37 & 0.00 & 0.00 \\ 
  GO:0032879 & regulation of localization & 1499 &  89 & 44.39 & 0.00 & 0.00 \\ 
  GO:0009306 & protein secretion & 311 &  34 & 9.21 & 0.00 & 0.00 \\ 
  GO:0051270 & regulation of cellular component movemen... & 494 &  44 & 14.63 & 0.00 & 0.00 \\ 
  GO:2000147 & positive regulation of cell motility & 264 &  31 & 7.82 & 0.00 & 0.00 \\ 
  GO:0022407 & regulation of cell-cell adhesion & 249 &  30 & 7.37 & 0.00 & 0.00 \\ 
  GO:0097530 & granulocyte migration &  66 &  16 & 1.95 & 0.00 & 0.00 \\ 
  GO:0012501 & programmed cell death & 1290 &  80 & 38.20 & 0.00 & 0.00 \\ 
  GO:1990266 & neutrophil migration &  57 &  15 & 1.69 & 0.00 & 0.00 \\ 
  GO:0002696 & positive regulation of leukocyte activat... & 190 &  26 & 5.63 & 0.00 & 0.00 \\ 
  GO:0006915 & apoptotic process & 1267 &  79 & 37.52 & 0.00 & 0.00 \\ 
  GO:0002252 & immune effector process & 461 &  42 & 13.65 & 0.00 & 0.00 \\ 
  GO:0032502 & developmental process & 3284 & 153 & 97.26 & 0.00 & 0.00 \\ 
  GO:0032940 & secretion by cell & 559 &  47 & 16.55 & 0.00 & 0.00 \\ 
  GO:0009617 & response to bacterium & 253 &  30 & 7.49 & 0.00 & 0.00 \\ 
  GO:0051272 & positive regulation of cellular componen... & 270 &  31 & 8.00 & 0.00 & 0.00 \\ 
   \hline
\end{tabular}
\end{table}

  
  Cluster 4, MF, Mm.
  Overrepresented GO molecular function terms among mouse genes which are highly abundant on day 7 p.i.. 
  There is overall no clear difference between pre infection and days 3 or 5 among these genesi. A slight
  tendency to higher abundance in second infection samples can be seen among these genes.\\
  % latex table generated in R 3.2.3 by xtable 1.8-2 package
% Fri Jun  3 15:14:08 2016
\begin{table}[ht]
\centering
\begin{tabular}{llrrrrr}
  \hline
GO.ID & Term & Annotated & Significant & Expected & p.value & adj.p \\ 
  \hline
GO:0005515 & protein binding & 5370 & 227 & 156.32 & 0.00 & 0.00 \\ 
  GO:0060089 & molecular transducer activity & 621 &  52 & 18.08 & 0.00 & 0.00 \\ 
  GO:0005125 & cytokine activity &  79 &  17 & 2.30 & 0.00 & 0.00 \\ 
  GO:0004871 & signal transducer activity & 534 &  45 & 15.54 & 0.00 & 0.00 \\ 
  GO:0004872 & receptor activity & 448 &  40 & 13.04 & 0.00 & 0.00 \\ 
  GO:0005102 & receptor binding & 851 &  58 & 24.77 & 0.00 & 0.00 \\ 
  GO:0038023 & signaling receptor activity & 361 &  33 & 10.51 & 0.00 & 0.00 \\ 
  GO:0004896 & cytokine receptor activity &  56 &  12 & 1.63 & 0.00 & 0.00 \\ 
  GO:0004888 & transmembrane signaling receptor activit... & 294 &  27 & 8.56 & 0.00 & 0.00 \\ 
  GO:0005126 & cytokine receptor binding & 144 &  18 & 4.19 & 0.00 & 0.00 \\ 
  GO:0003823 & antigen binding &  58 &  11 & 1.69 & 0.00 & 0.00 \\ 
  GO:0016493 & C-C chemokine receptor activity &   6 &   4 & 0.17 & 0.00 & 0.00 \\ 
  GO:0008009 & chemokine activity &  19 &   6 & 0.55 & 0.00 & 0.00 \\ 
  GO:0042379 & chemokine receptor binding &  28 &   7 & 0.82 & 0.00 & 0.00 \\ 
  GO:0001637 & G-protein coupled chemoattractant recept... &  12 &   5 & 0.35 & 0.00 & 0.00 \\ 
  GO:0004950 & chemokine receptor activity &  12 &   5 & 0.35 & 0.00 & 0.00 \\ 
  GO:0005085 & guanyl-nucleotide exchange factor activi... & 155 &  15 & 4.51 & 0.00 & 0.00 \\ 
  GO:0005509 & calcium ion binding & 273 &  21 & 7.95 & 0.00 & 0.00 \\ 
  GO:0001948 & glycoprotein binding &  74 &  10 & 2.15 & 0.00 & 0.00 \\ 
  GO:0098772 & molecular function regulator & 792 &  43 & 23.05 & 0.00 & 0.00 \\ 
  GO:0008528 & G-protein coupled peptide receptor activ... &  25 &   6 & 0.73 & 0.00 & 0.00 \\ 
  GO:0042605 & peptide antigen binding &  25 &   6 & 0.73 & 0.00 & 0.00 \\ 
  GO:0001653 & peptide receptor activity &  26 &   6 & 0.76 & 0.00 & 0.00 \\ 
  GO:0005488 & binding & 8136 & 268 & 236.84 & 0.00 & 0.00 \\ 
  GO:0048365 & Rac GTPase binding &  38 &   7 & 1.11 & 0.00 & 0.00 \\ 
  GO:0050839 & cell adhesion molecule binding & 112 &  12 & 3.26 & 0.00 & 0.00 \\ 
  GO:0019956 & chemokine binding &  10 &   4 & 0.29 & 0.00 & 0.00 \\ 
  GO:0004930 & G-protein coupled receptor activity & 103 &  11 & 3.00 & 0.00 & 0.00 \\ 
  GO:0004908 & interleukin-1 receptor activity &   5 &   3 & 0.15 & 0.00 & 0.00 \\ 
  GO:0019957 & C-C chemokine binding &   5 &   3 & 0.15 & 0.00 & 0.00 \\ 
  GO:1990782 & protein tyrosine kinase binding &  44 &   7 & 1.28 & 0.00 & 0.00 \\ 
  GO:0046983 & protein dimerization activity & 797 &  41 & 23.20 & 0.00 & 0.00 \\ 
  GO:0019899 & enzyme binding & 1457 &  64 & 42.41 & 0.00 & 0.00 \\ 
  GO:0019900 & kinase binding & 537 &  30 & 15.63 & 0.00 & 0.00 \\ 
  GO:0005544 & calcium-dependent phospholipid binding &  25 &   5 & 0.73 & 0.00 & 0.00 \\ 
  GO:0005088 & Ras guanyl-nucleotide exchange factor ac... & 101 &  10 & 2.94 & 0.00 & 0.00 \\ 
  GO:0005415 & nucleoside:sodium symporter activity &   2 &   2 & 0.06 & 0.00 & 0.00 \\ 
  GO:0035717 & chemokine (C-C motif) ligand 7 binding &   2 &   2 & 0.06 & 0.00 & 0.00 \\ 
  GO:0071791 & chemokine (C-C motif) ligand 5 binding &   2 &   2 & 0.06 & 0.00 & 0.00 \\ 
  GO:0042277 & peptide binding & 145 &  12 & 4.22 & 0.00 & 0.00 \\ 
  GO:0030971 & receptor tyrosine kinase binding &  42 &   6 & 1.22 & 0.00 & 0.00 \\ 
  GO:0001664 & G-protein coupled receptor binding & 134 &  11 & 3.90 & 0.00 & 0.00 \\ 
  GO:0019901 & protein kinase binding & 485 &  26 & 14.12 & 0.00 & 0.00 \\ 
  GO:0044877 & macromolecular complex binding & 1139 &  50 & 33.16 & 0.00 & 0.00 \\ 
  GO:0005178 & integrin binding &  63 &   7 & 1.83 & 0.00 & 0.00 \\ 
  GO:0032403 & protein complex binding & 757 &  36 & 22.04 & 0.00 & 0.00 \\ 
  GO:0005543 & phospholipid binding & 248 &  16 & 7.22 & 0.00 & 0.00 \\ 
  GO:0004859 & phospholipase inhibitor activity &   3 &   2 & 0.09 & 0.00 & 0.00 \\ 
  GO:0019834 & phospholipase A2 inhibitor activity &   3 &   2 & 0.09 & 0.00 & 0.00 \\ 
  GO:0031726 & CCR1 chemokine receptor binding &   3 &   2 & 0.09 & 0.00 & 0.00 \\ 
  GO:0034988 & Fc-gamma receptor I complex binding &   3 &   2 & 0.09 & 0.00 & 0.00 \\ 
  GO:0035662 & Toll-like receptor 4 binding &   3 &   2 & 0.09 & 0.00 & 0.00 \\ 
  GO:0033218 & amide binding & 164 &  12 & 4.77 & 0.00 & 0.01 \\ 
  GO:0016209 & antioxidant activity &  50 &   6 & 1.46 & 0.00 & 0.01 \\ 
  GO:0008289 & lipid binding & 425 &  23 & 12.37 & 0.00 & 0.01 \\ 
  GO:0005149 & interleukin-1 receptor binding &  11 &   3 & 0.32 & 0.00 & 0.01 \\ 
  GO:0048020 & CCR chemokine receptor binding &  11 &   3 & 0.32 & 0.00 & 0.01 \\ 
  GO:0003779 & actin binding & 263 &  16 & 7.66 & 0.00 & 0.01 \\ 
  GO:0046977 & TAP binding &  12 &   3 & 0.35 & 0.00 & 0.01 \\ 
  GO:0016175 & superoxide-generating NADPH oxidase acti... &   4 &   2 & 0.12 & 0.00 & 0.01 \\ 
  GO:0050786 & RAGE receptor binding &   4 &   2 & 0.12 & 0.00 & 0.01 \\ 
  GO:0031267 & small GTPase binding & 220 &  14 & 6.40 & 0.01 & 0.01 \\ 
  GO:0030881 & beta-2-microglobulin binding &  13 &   3 & 0.38 & 0.01 & 0.01 \\ 
  GO:0048306 & calcium-dependent protein binding &  41 &   5 & 1.19 & 0.01 & 0.01 \\ 
  GO:0046982 & protein heterodimerization activity & 323 &  18 & 9.40 & 0.01 & 0.01 \\ 
  GO:0017048 & Rho GTPase binding &  77 &   7 & 2.24 & 0.01 & 0.01 \\ 
  GO:0001968 & fibronectin binding &  14 &   3 & 0.41 & 0.01 & 0.01 \\ 
  GO:0004435 & phosphatidylinositol phospholipase C act... &  14 &   3 & 0.41 & 0.01 & 0.01 \\ 
  GO:0042608 & T cell receptor binding &  14 &   3 & 0.41 & 0.01 & 0.01 \\ 
  GO:0050750 & low-density lipoprotein particle recepto... &  14 &   3 & 0.41 & 0.01 & 0.01 \\ 
  GO:0071813 & lipoprotein particle binding &  14 &   3 & 0.41 & 0.01 & 0.01 \\ 
  GO:0071814 & protein-lipid complex binding &  14 &   3 & 0.41 & 0.01 & 0.01 \\ 
  GO:0008081 & phosphoric diester hydrolase activity &  59 &   6 & 1.72 & 0.01 & 0.01 \\ 
  GO:0005539 & glycosaminoglycan binding &  78 &   7 & 2.27 & 0.01 & 0.01 \\ 
  GO:0008201 & heparin binding &  60 &   6 & 1.75 & 0.01 & 0.01 \\ 
  GO:0003933 & GTP cyclohydrolase activity &   5 &   2 & 0.15 & 0.01 & 0.01 \\ 
  GO:0035325 & Toll-like receptor binding &   5 &   2 & 0.15 & 0.01 & 0.01 \\ 
  GO:0045236 & CXCR chemokine receptor binding &   5 &   2 & 0.15 & 0.01 & 0.01 \\ 
  GO:0030234 & enzyme regulator activity & 594 &  28 & 17.29 & 0.01 & 0.01 \\ 
  GO:0038024 & cargo receptor activity &  28 &   4 & 0.82 & 0.01 & 0.01 \\ 
  GO:0005044 & scavenger receptor activity &  15 &   3 & 0.44 & 0.01 & 0.01 \\ 
  GO:0070325 & lipoprotein particle receptor binding &  15 &   3 & 0.44 & 0.01 & 0.01 \\ 
   \hline
\end{tabular}
\end{table}

% figure 1
%\begin{figure}[h]
%\begin{center}
%\includegraphics[width=0.8\textwidth]{distributions_mouseNocutoff} % Include the image placeholder.png
%\caption{Density distribution of raw transcript counts (log10) for all
%  mouse samples in analysis. All samples share the same bimodal
%  distribution trend by visual inspection. The first density peak
%  occur at $1 - 10^2$ transcripts, \textit{i.e.} transcripts which are
%  detected between 1 and 100 times in that sample. The second density
%  peak occurs in the range of $10^3 - 10^4$.}
%\end{center}
%\end{figure}

%----------------------------------------------------------------------------------------
%	SECTION 5
%----------------------------------------------------------------------------------------

\section{Conclusions}

Mouse\\

Parasite\\


%----------------------------------------------------------------------------------------
%	SECTION 6
%----------------------------------------------------------------------------------------

%\section{Answers to Definitions}

%\begin{enumerate}
%\begin{item}
%  The \emph{atomic weight of an element} is the relative weight of
%  one of its atoms compared to C-12 with a weight of
%  12.0000000$\ldots$, hydrogen with a weight of 1.008, to oxygen with
%  a weight of 16.00. Atomic weight is also the average weight of all
%  the atoms of that element as they occur in nature.
%\end{item}
%\begin{item}
%  The \emph{units of atomic weight} are two-fold, with an identical
%  numerical value. They are g/mole of atoms (or just g/mol) or
%  amu/atom.
%\end{item}
%\begin{item}
%  \emph{Percentage discrepancy} between an accepted (literature)
%  value and an experimental value is
%\begin{equation*}
%\frac{\mathrm{experimental\;result} - \mathrm{accepted\;result}}{\mathrm{accepted\;result}}
%\end{equation*}
%\end{item}
%\end{enumerate}

%----------------------------------------------------------------------------------------
%	BIBLIOGRAPHY
%----------------------------------------------------------------------------------------

%\bibliographystyle{apalike}

%\bibliography{sample}

%----------------------------------------------------------------------------------------


\end{document}
